\documentclass[12pt, titlepage]{article}

\usepackage{booktabs}
\usepackage{tabularx}
\usepackage{hyperref}
\hypersetup{
    colorlinks,
    citecolor=black,
    filecolor=black,
    linkcolor=red,
    urlcolor=blue
}
\usepackage[round]{natbib}

%% Comments

\usepackage{color}

\newif\ifcomments\commentstrue

\ifcomments
\newcommand{\authornote}[3]{\textcolor{#1}{[#3 ---#2]}}
\newcommand{\todo}[1]{\textcolor{red}{[TODO: #1]}}
\else
\newcommand{\authornote}[3]{}
\newcommand{\todo}[1]{}
\fi

\newcommand{\wss}[1]{\authornote{blue}{SS}{#1}}
\newcommand{\als}[1]{\authornote{magenta}{AS}{#1}}


\begin{document}

\title{Test Report: Runge-Kutta Generator} 
\author{Alexander Schaap}
\date{\today}
	
\maketitle

\pagenumbering{roman}

\section{Revision History}

The latest version can be found at \url{https://github.com/aschaap/cas741}.\\

\begin{tabularx}{\textwidth}{p{3cm}p{2cm}X}
\toprule {\bf Date} & {\bf Version} & {\bf Notes}\\
\midrule
December 18 & 1.0 & Initial (and final) version\\
%Date 2 & 1.1 & Notes\\
\bottomrule
\end{tabularx}

~\newpage

\section{Symbols, Abbreviations and Acronyms}
See the \href{../SRS/CA.pdf#ssec:symbols}{Table of Symbols} in the SRS at 
\url{https://github.com/aschaap/cas741}.
\renewcommand{\arraystretch}{1.2}
\begin{tabular}{l l} 
  \toprule		
  \textbf{symbol} & \textbf{description}\\
  \midrule 
  T & Test\\
  \bottomrule
\end{tabular}\\

\wss{symbols, abbreviations or acronyms -- you can reference the SRS tables if needed}

\newpage

\tableofcontents

%\listoftables %if appropriate
%
%\listoffigures %if appropriate

\newpage

\pagenumbering{arabic}

\section{Introduction}
This document will summarize whether the implementation satisfies the 
requirements implied in the \href{../SRS/CA.pdf}{Commonality Analysis} 
available at \url{https://github.com/aschaap/cas741}.

\section{Functional Requirements Evaluation}
The implementation contains both the RK2 and RK4 methods, and allows users to 
choose between them. It generates a function that will return numerical 
approximations for a given RK method, ODE, interval, step size, and initial 
condition.

\section{Nonfunctional Requirements Evaluation}
Since this is a library, the primary nonfunctional requirement is performance.

\subsection{Usability}
Usability was not a goal, but the implementation ensures the exposed function 
interface is minimalistic.

\subsection{Performance}
Performance evaluation has been moved to phase 2, and will therefore be added 
later.
	
\section{Comparison to Existing Implementation}	
Comparing the output of the system tests to the same equation in WolframAlpha 
shows that this implementation is close for slopes that are not too steep.
The infinite values returned for the higher order ODE are also accurate, since 
the function appears undefined at these points on the interval. For the stiff 
ODE, the results are predictably all over the place.

\section{Unit Testing}
The OUnit framework available for OCaml (and by extension, MetaOCaml) has been 
employed to run some unit tests.

\section{Changes Due to Testing}
N/A.

\section{Automated Testing}
GNU Make has been employed to both run the unit tests and generate output for 
the user to manually examine.

\section{Trace to Requirements}
N/A. There are no explicit requirements, since a 
\href{../SRS/CA.pdf}{Commonality Analysis} has been done rather than a 
requirements specification.

\section{Trace to Modules}
N/A. This library has been decomposed according to binding times, rather than 
the traditional modularization. See the \href{../Design/Design.pdf}{Design 
document} available at \url{https://github.com/aschaap/cas741} for more details 
and justification for this.

\section{Code Coverage Metrics}
Due to time constraints, this has also been moved to phase 2.

\bibliographystyle{plainnat}

\bibliography{SRS}

\end{document}