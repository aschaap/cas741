\documentclass{article}

\usepackage[margin=3.8cm]{geometry}
\usepackage{tabularx}
\usepackage{booktabs}
\usepackage{hyperref}

\title{CAS 741: Problem Statement\\rkf45.ml}

\author{Alexander Schaap (schaapal)}

\date{September 18, 2017}

%% Comments

\usepackage{color}

\newif\ifcomments\commentstrue

\ifcomments
\newcommand{\authornote}[3]{\textcolor{#1}{[#3 ---#2]}}
\newcommand{\todo}[1]{\textcolor{red}{[TODO: #1]}}
\else
\newcommand{\authornote}[3]{}
\newcommand{\todo}[1]{}
\fi

\newcommand{\wss}[1]{\authornote{blue}{SS}{#1}}
\newcommand{\als}[1]{\authornote{magenta}{AS}{#1}}


\begin{document}

\maketitle

\begin{table}[hp]
\caption{Revision History} \label{TblRevisionHistory}
\begin{tabularx}{\textwidth}{llX}
\toprule
\textbf{Date} & \textbf{Developer(s)} & \textbf{Change}\\
\midrule
Sept 18 & Alex & Addressed Dr.~Smith's feedback; toned down metaprogramming,
removed RKF45, emphasized ODE solver family library\\
\midrule
Sept 15 & Alex & Initial version\\
\bottomrule
\end{tabularx}
\end{table}

%Put your problem statement here.  Comments to you can be added, like this:
%
%\wss{comment}
%
%You can also leave comments for yourself, like this:
%
%\als{comment}

Scientists in many areas use ordinary differential equations (ODEs) in their
research. For this reason, the goal of this project is to create a family of ODE
solvers in the form of a reusable library.

While many solvers exist already, the emphasis on reuse hopefully spares
scientists the effort of reimplementing these functions over and over, along
with the associated increased chance of software bugs. Additionally, a new
programming paradigm may both save time while creating the program family as
well as provide performance benefits over existing
approaches.\footnote{\url{https://arxiv.org/pdf/1612.06668}}

Multi-stage programming (MSP) is a metaprogramming paradigm.
%in which type-safe code fragments are created and combined.
Its goal is to enable development of ``generic software that does not pay a
runtime penalty for this
generality''\footnote{\url{https://web.archive.org/web/20170804032218/http://www.cs.rice.edu/~taha/publications/journal/dspg04a.pdf}}.
%(Note that this heavily implies a compile-time penalty.)
MSP allows programmers to generate entire product families once all possible
variabilities have been accounted for. It also has the potential for
%enables potentially every bit of information known at compile-time to be
%optimized out of the generated code. Doing this can lead to significant
performance benefits, which ultimately allow users of programs constructed in
this way to focus more on their other activities.

%A popular method for numerical approximation of ODEs is the
%Runge-Kutta-Fehlberg (RKF45) method. (The 4 in the abbreviation comes from the
%method being of order O($h^4$) and the 5 signifies the error estimator of order
%O($h^5$). This method has been implemented in a multi-stage programming
%language called MetaOCaml by Dr.~J.~Carette. \wss{put a slash (or tilde) after
%the period to get the spacing right. That is, it should be Dr.\ J.\ Carette.
%Otherwise LaTeX interprets a period as the end of a sentence and leaves two
%spaces.} \als{Done, but it seems this section is focused too much on how rather
%than what.} The goal therefore would be to recreate and improve this
%implementation.

One might argue that manually optimizing for performance is a thing of the past.
While
%it is true that
computer hardware has become much faster,
%there are still computationally intensive problems as well as
a large portion of the world population that cannot afford the latest and
greatest hardware
%. This could be motivated by financial reasons as well as
and/or has a need for (less powerful) mobile hardware.
%(e.g. low-voltage processors in laptops)

Primary stakeholders would be scientists using this implementation either
directly or indirectly, as well as programmers integrating this implementation
into larger pieces of software.

The environment of the software is limited to what the OCaml compiler supports,
but this includes Windows, macOS and Linux.

\wss{Please format your tex file with 80 width lines of text, with a hard
return at the end of each line.}
\als{Done, except for the URL.}

\end{document}\grid