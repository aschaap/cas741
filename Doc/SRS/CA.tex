\documentclass[12pt]{article}

\usepackage{amsmath, mathtools}
\usepackage{amsfonts}
\usepackage{amssymb}
\usepackage{graphicx}
\usepackage{colortbl}
\usepackage{xr}
\usepackage[destlabel]{hyperref}%inter-PDF linking with labels
\usepackage{longtable}
\usepackage{xfrac}
\usepackage{tabularx}
\usepackage{float}
\usepackage{siunitx}
\usepackage{booktabs}
\usepackage{caption}
\usepackage{pdflscape}
\usepackage{afterpage}
\usepackage{listings}

\usepackage[round]{natbib}

%\usepackage{refcheck}

\hypersetup{
%    bookmarks=true,         % show bookmarks bar?
      colorlinks=true,       % false: boxed links; true: colored links
    linkcolor=red,          % color of internal links (change box color with linkbordercolor)
    citecolor=green,        % color of links to bibliography
    filecolor=magenta,      % color of file links
    urlcolor=cyan           % color of external links
}

%% Comments

\usepackage{color}

\newif\ifcomments\commentstrue

\ifcomments
\newcommand{\authornote}[3]{\textcolor{#1}{[#3 ---#2]}}
\newcommand{\todo}[1]{\textcolor{red}{[TODO: #1]}}
\else
\newcommand{\authornote}[3]{}
\newcommand{\todo}[1]{}
\fi

\newcommand{\wss}[1]{\authornote{blue}{SS}{#1}}
\newcommand{\als}[1]{\authornote{magenta}{AS}{#1}}


% For easy change of table widths
\newcommand{\colZwidth}{1.0\textwidth}
\newcommand{\colAwidth}{0.13\textwidth}
\newcommand{\colBwidth}{0.82\textwidth}
\newcommand{\colCwidth}{0.1\textwidth}
\newcommand{\colDwidth}{0.05\textwidth}
\newcommand{\colEwidth}{0.8\textwidth}
\newcommand{\colFwidth}{0.17\textwidth}
\newcommand{\colGwidth}{0.5\textwidth}
\newcommand{\colHwidth}{0.28\textwidth}

% Used so that cross-references have a meaningful prefix
\newcounter{defnum} %Definition Number
\newcommand{\dthedefnum}{GD\thedefnum}
\newcommand{\dref}[1]{GD\ref{#1}}
\newcounter{datadefnum} %Datadefinition Number
\newcommand{\ddthedatadefnum}{DD\thedatadefnum}
\newcommand{\ddref}[1]{DD\ref{#1}}
\newcounter{theorynum} %Theory Number
\newcommand{\tthetheorynum}{T\thetheorynum}
\newcommand{\tref}[1]{T\ref{#1}}
\newcounter{tablenum} %Table Number
\newcommand{\tbthetablenum}{T\thetablenum}
\newcommand{\tbref}[1]{TB\ref{#1}}
\newcounter{assumpnum} %Assumption Number
\newcommand{\atheassumpnum}{P\theassumpnum}
\newcommand{\aref}[1]{A\ref{#1}}
\newcounter{goalnum} %Goal Number
\newcommand{\gthegoalnum}{P\thegoalnum}
\newcommand{\gsref}[1]{GS\ref{#1}}
\newcounter{instnum} %Instance Number
\newcommand{\itheinstnum}{IM\theinstnum}
\newcommand{\iref}[1]{IM\ref{#1}}
\newcounter{reqnum} %Requirement Number
\newcommand{\rthereqnum}{P\thereqnum}
\newcommand{\rref}[1]{R\ref{#1}}
\newcounter{lcnum} %Likely change number
\newcommand{\lthelcnum}{LC\thelcnum}
\newcommand{\lcref}[1]{LC\ref{#1}}

\newcommand{\famname}{RK Generator} % PUT YOUR PROGRAM NAME HERE

\usepackage{fullpage}
\usepackage{tikz}
\usetikzlibrary{arrows}

\begin{document}

\title{Runge-Kutta (RK) Generator} 
\author{Alexander Schaap}
\date{\today}

\maketitle

~\newpage

\pagenumbering{roman}

\section{Revision History}
The latest version can be found at \url{https://github.com/aschaap/cas741}.\\

\noindent
\begin{tabularx}{\textwidth}{p{3.5cm}p{2cm}X}
\toprule {\bf Date} & {\bf Version} & {\bf Notes}\\
\midrule
October 20, 2017 & 1.0 & Initial version\\
December 17, 2017 & 1.1 & Addressed feedback\\
\today & 1.2 & Final version\\
\bottomrule
\end{tabularx}

~\newpage
	
\section{Reference Material}

This section records information for easy reference.

\subsection{Table of Units}

Since the application of this software is dependent on the program using the 
functions provided by this program family, no physical units are used 
throughout this document.

\subsection{Table of Symbols}\label{ssec:symbols}

The table that follows summarizes the symbols used in this document. As stated 
above, no physical units are associated with any of these.  The choice of 
symbols was made to be consistent with the 
literature describing ODEs.\footnote{Specifically, 
\cite{corless_graduate_2013}.}  The symbols are listed in alphabetical order.

\renewcommand{\arraystretch}{1.2}
%\noindent \begin{tabularx}{1.0\textwidth}{l l X}
\noindent \begin{longtable*}{l p{12cm}} \toprule
\textbf{symbol} & \textbf{description}\\
\midrule 
$\mathbb{C}$ & Complex numbers
\\ 
${\bf f}$ & ODE for which to solve
\\
$h$ & step size
\\ 
${\bf k}_{1},{\bf k}_{2}, {\bf k}_{3}, {\bf k}_{4}$ & intermediate 
variables in the RK4 method, similar to 
${\bf Y}_1$ \wss{I don't know what the subscript notation means here.  Do you
  have 4 vectors of $k$ values or one vector with 4 values?} \als{I meant 4 
  vectors, but I've tried to make it less ambiguous.}
\\ 
$\mathbb{R}$ & Real numbers
\\ 
$t$ & an independent variable, commonly time
\\ 
$t_0$ & beginning of interval
\\ 
$t_N$ & end of interval
\\ 
$t_k$ & any $t$ on the interval $[t_0..t_N]$
\\ 
${\bf x(t)}$ & a function of $t$
\\ 
${\bf x}'$ & derivative of ${\bf x(t)}$ with respect to $t$ \wss{add: with 
respect to $t$}\als{Done.}
\\ 
${\bf x}_0$ & initial values for ${\bf f}$
\\ 
${\bf x}_k$ & (numerical approximation of) a point on the solution for 
${\bf f}$
\\ 
${\bf Y}_1$ & intermediate variables in the RK2 method, similar to ${\bf 
k}_{1}..{\bf k}_{4}$
\\
\bottomrule
\end{longtable*}

\subsection{Abbreviations and Acronyms}

\renewcommand{\arraystretch}{1.2}
\begin{tabular}{l l} 
  \toprule
  \textbf{symbol} & \textbf{description}\\
  \midrule 
  A & Assumption\\
  DD & Data Definition\\
  GD & General Definition\\
  GS & Goal Statement\\
  IM & Instance Model\\
  LC & Likely Change\\
  ODE & Ordinary Differential Equation\\
  PS & Physical System Description\\
  R & Requirement\\
  RK & Runge-Kutta\\
  RK2 & Second order Runge-Kutta method\\
  RK4 & Fourth order Runge-Kutta method\\
%  RKF45 & Runge-Kutta-Fehlberg method\\
  \famname{} & Family of programs based on the RK2 / RK4 method(s)\\
  SRS & Software Requirements Specification\\
  STEM & Science, Technology, Engineering \& Mathematics\\
  T & Theoretical Model\\
  \bottomrule
\end{tabular}\\

\newpage

\tableofcontents

~\newpage

\pagenumbering{arabic}

\section{Introduction}
%modelled after 3dfim+
This document provides an overview of the commonality analysis (CA) for the 
\famname{} program family. Members of program family are produced by a code 
generator. Generated members provide numerical approximations for given 
ordinary differential equations (ODEs) using Runge-Kutta (RK) methods. Most of 
the calculations happens during code generation, producing a different family 
member for each given combination of RK method, ODE, interval, step size, and 
initial values. 
%The family member presents the results in the form of a function that can be 
%called. 
The current section describes the purpose of this document, the scope of this 
family, the organization of the remainder of the document and the 
characteristics of the intended reader.

\subsection{Purpose of Document}
%modelled after 3dfim+
The main purpose of this document is to provide sufficient information to 
understand what the \famname{} is expected to do.
The goals and theoretical models used in the \famname{} are provided, as are 
assumptions and unambiguous definitions.
This document should ultimately aid the development of the software by 
specifying and constraining the design, and by providing a starting point for 
the creation of test cases.
It should also give developers maintaining and/or improving the software a 
better idea of what it should and should not do, though improvement could 
require them to update this document.
For users (i.e. developers looking to incorporate this into their own 
software), this document details the uses and limitations of the \famname{}.
\nameref{ssec:intended-reader} covers the intended audience in more detail.

\subsection{Scope of the Family}
%loosely modelled after 3dfim+
The responsibility of the family is solving the provided first order ODEs (in 
explicit form) for given intervals, initial values and uniform step sizes using 
the specified RK method. The 
user will have to ensure the RK method specified is appropriate for their 
particular problem(s), and they will have to ensure appropriate and correct 
input is provided (see \nameref{sec_datadef}). 
Additionally, the user will also have to process the output appropriately; 
namely use the generated function that will take inputs within the given 
interval and provide approximate solutions for them.
\wss{Your scope should specify which kind of ODEs you solve.  Your software
  solves systems of first order ODEs.  You do not solve higher order ODEs,
  presumably because they can be recast as systems of first order ODEs.  In
  addition, the functional form of your ODE can be given explicitly, as opposed
  to being in implicit form.}
\als{I've added this.}

\subsection{Characteristics of Intended Reader}\label{ssec:intended-reader} 
The reader is expected to have some undergraduate background in a math-heavy 
science or engineering, as well as a graduate understanding of code generation. 
Ideally, they have been 
exposed to some undergraduate calculus courses that covered numerical methods 
for ODE's.
Programming courses would also help, especially a graduate understanding of 
functional programming and metaprogramming. 
%Metaprogramming may be beyond the scope of most undergraduate degrees though.
\wss{For the characteristics of intended reader try to be
  more specific about the education.  What degree?  What course areas?  What
  level?}
\als{I've tried to add some detail.}

\subsection{Organization of Document}
%inspired by FamilyOfMaterialModels
The organization of this document follows the template for a commonality 
analysis (CA) \wss{The citations are enough; you
  don't need to mention me separately} \als{Removed your name.} adapted from 
\cite{Lai2004}, \cite{SmithAndLai2005}, \cite{Smith2006} and 
\cite{SmithEtAl2007}.

The structure of this document is essentially top-down, starting with a 
birds-eye view with detail added as 
one proceeds through it. A general system description is provided first, 
consisting of context, user characteristics and system constraints,
followed by commonalities and variabilities. The commonalities section includes 
some background information, terminology and definitions, data definitions, 
goals and theoretical models. Variabilities include assumptions, different ways 
of calculating the result, and how output may (not) differ between variants.
Traceability matrices and references are listed at the end.

\section{General System Description}

This section identifies the interfaces between the system and its environment,
describes the potential user characteristics and lists the potential system
constraints.

\subsection{Potential System Contexts}

\begin{figure}[htb]
    \centering
\begin{tikzpicture} [
    auto,
    c/.style = {circle, draw, text centered},
    b/.style = {rectangle, draw, text centered, rounded corners, minimum 
    height=2em},
    arrow/.style = {draw, ->},
    node distance = 3cm
    ]
    \node[c](user) {Parent program};
    \node[b, right of=user, node distance = 8cm](RKgen) {\famname{}}; 
    \node[b, below of=RKgen](gen) {Generated code};
    
    \draw[arrow] (user) -- node {Calls with inputs} (RKgen);
    \draw[arrow] (RKgen) -- node {Produces} (gen);
    \draw[arrow] (user) -- node {Calls with input} (gen);
    \draw[arrow] (gen) -| node {Results} (user);
    
\end{tikzpicture}
\caption{System Context}
\label{fig:systemcontext}
\end{figure}

\autoref{fig:systemcontext} shows the system context. A circle represents than 
external entity outside the software, which is the ``user'', or parent program 
making use of the generator as a library. A rectangle is used to represent the 
generator as well as the \famname{} family members it can generate. 

Using the library is as simple as using an \lstinline[language=ML]|open| 
statement, though even this is not strictly necessary. If it is used, it will 
be compiled into the parent program.
Ideally, the parent program knows the ODE(s) and other inputs at compile time, 
and provides them to the generator when calling it to generate family members.
The parent program subsequently uses each generated family member many times at 
run-time for different points within the interval, to maximize the benefit of 
code generation.

Alternatively, the ODE(s) or other inputs are not known at compile time; the 
generator can still be called at run-time, but hopefully the parent program can 
offset the incurred cost of this on-the-fly generation by running the optimized 
generated code enough times.

A breakdown of responsibilities of both the user and the \famname{} is provided 
below:
\begin{itemize}
\item User Responsibilities:
\begin{itemize}
\item Provide appropriate input to the system
\item Ensure assumptions on the input are met (see \autoref{ssec:Assumptions})
\item Correctly process the resulting output function
\end{itemize}
\item \famname{} Responsibilities:
\begin{itemize}
\item Calculate the correct output for the given input
\end{itemize}
\end{itemize}

\subsection{Potential User Characteristics} \label{SecUserCharacteristics}

The most common user of \famname{} will be other programs. However, one or more 
programmers are needed to write the code that calls the functions provided by 
this family.
These programmers therefore should have an understanding of ODEs, the accuracy 
of numerical approximation (using RK methods) and OCaml programming. Knowledge 
of MetaOCaml should not be necessary 
if the program family is able to hide its usage thereof sufficiently.

\subsection{Potential System Constraints}

The responsibility of generating family members in a type-safe way restricts 
the system to the MetaOCaml extension of the OCaml programming language.
Code generation errors can be tricky (think C++ templates), so using a 
type-safe language ensures that if it compiles, it most likely works as 
expected.
This also helps avoid more subtle errors that could easily go unnoticed.
Alternatives tend to make less guarantees about the generated code.
\section{Commonalities}

This section contains the commonalities between the \famname{} family members. 
This section starts with succinct background information about the problem, 
followed by terminology and definitions, data definitions, goal statements and 
ending with theoretical models. 

\subsection{Background Overview} \label{Sec_Background}

There are various numerical methods for approximating ordinary differential 
equation (ODE) given initial values. This problem is called an initial value 
problem (IVP, see \tref{T_IVP}). A well-known way to solve these problems is 
through Runge-Kutta (RK) methods.% (which include Euler's method).

To explain RK methods, it is easiest to start with the Euler method. Euler's 
method consists of simply taking the slope of the current point, and 
extrapolating that for a distance of step size $h$ on the x-axis to create a 
new point. Since the slope of the point changes, this method creates an 
approximation that will always seem to ``lag behind'' the actual ODE, but with 
a small enough step size, it should be pretty close. For larger intervals, 
you'll see that the error appears to accumulate. See
\autoref{fig:euler-method} as a simple illustration.

\begin{figure}[htb]
  \centering
  \def\svgwidth{0.5\textwidth}
  \input{figs/euler-method.pdf_tex}
  \caption{Approximation obtained using Euler's method versus the actual 
  solution}
  \label{fig:euler-method}
\end{figure}

Using Euler's method as a starting point, one can see how accuracy could be 
improved by 
taking the slopes at both the current point as well as the current point 
incremented by step 
size $h$, and using the average of those two slopes as the slope with which to 
extrapolate to approximate the point that is step size $h$ away from the 
current 
point. This is the second order Runge-Kutta (RK2) method (for the mathematical 
definition, see \iref{IM_RK2}).

Runge-Kutta methods are a collection of methods that use slopes of multiple, 
sometimes intermediate, points (found through successive extrapolation) to 
create more accurate approximations. The slopes are also assigned potentially 
different weights when finding the average. The more points one uses (the 
higher the order of the RK method), the smaller the error should be (the error 
will always be within $h^{(order\ of\ used\ RK\ method\ +\ 1)}$, which becomes
smaller quite fast for small step sizes).

\subsection{Terminology and  Definitions}

This subsection provides a list of terms that are used in the subsequent
sections and their meaning, with the purpose of reducing ambiguity and making it
easier to correctly understand the requirements:

\begin{itemize}

\item Code generation: producing code programmatically.

\item Compile-time: the time during which a program is compiled.

\item Continuous function: a function that evaluates to a value that is not 
infinity or undefined for any input.

\item Initial values (initial condition): a starting point from which the rest 
of the approximation 
can be calculated. (See \ddref{DD_initialvalues}.)

\item Initial value problem (IVP): an ODE along with initial values for which a 
numerical approximation must be found.

\item Interval: interval inside which the ODE needs to be approximated. (See 
\ddref{DD_interval}.)

\item MetaOCaml: an extension to the OCaml programming language which adds 
staging annotations that enable MSP.

\item Metaprogramming: programming software that takes in and/or produces 
programs.

\item Multi-stage programming (MSP): a form of metaprogramming where the 
execution of specific code fragments is delayed; these fragments are then 
combined into larger fragments and ultimately executed.

\item Numerical method: a specific way to numerically approximate solutions to 
a particular problem (rather than symbolic manipulations).

\item OCaml: A multi-paradigm programming language (imperative \& functional) 
with a syntax that will appear somewhat unusual to most.

\item Ordinary differential equation (ODE): an equation that involves some 
ordinary (rather than partial) derivatives of a function. The goal is to find 
the function that satisfies the equation, and this is not always easily 
achievable through integration. (See \ddref{DD_ODE}.)

\item Runge-Kutta methods: a set of methods that approximate ODEs in IVPs for a 
given interval and step size.

\item Run-time: the time during which a program is run.

\item Step size: the distance between any two points for which to find 
approximations; it essentially determines how many points there will be once 
the interval is known. Once these points have been approximated, one could 
interpolate so that the whole interval is solved for. (See also 
\ddref{DD_stepsize}.)

\item Stiff ODE: an ODE for which certain numerical methods are numerically 
unstable unless the step size is extremely small. There is no exact definition.

\item Type-safe: Applicable to programming languages; property that follows 
from such a language not allowing conversions/reductions of valid (adherent to 
the language's type system) programs to a state where the type system is 
violated, nor from where no further reduction is possible. So all valid 
programs can evaluate to a value in the language's set of evaluation rules. A 
simple example: whether the type system allows for integers to be used as 
floating point number.
\end{itemize}

\subsection{Data Definitions} \label{sec_datadef}

This section collects and defines all the data needed to build the instance
models. The dimension of each quantity is also given.
~\newline

\noindent
\begin{minipage}{\textwidth}
\renewcommand*{\arraystretch}{1.5}
\begin{tabular}{| p{\colAwidth} | p{\colBwidth}|}
\hline
\rowcolor[gray]{0.9}
Number& DD\refstepcounter{datadefnum}\thedatadefnum \label{DD_ODE}\\
\hline
Label& \bf Ordinary Differential Equation (ODE)\\
\hline
Symbol &${\bf f}$\\
\hline
 Units& N/A\\
 \hline
%  SI Units & \si{\watt\per\square\metre}\\
%  \hline
  Equation&${\bf x'} = {\bf f} (t,{\bf x} (t))$\\
  \hline
  Description & 
                 ${\bf f} : \mathbb{R} \times \mathbb{C}^n \rightarrow 
                 \mathbb{C}^n$ is the equation for which we ultimately want to 
                 find 
                 a numerical approximation. \wss{I don't think this is true.
                You know $f$ and you are trying to find $x$.}
  \\
  \hline
  Sources & \cite{corless_graduate_2013} \\
  \hline
  Ref.\ By & \iref{IM_RK4}, \iref{IM_RK2}, \tref{T_IVP}, 
  \aref{A_ODEnonstiffcontinuous}, \aref{A_initialvalues}\\
  \hline
\end{tabular}
\end{minipage}\\

~\newline

\noindent
\begin{minipage}{\textwidth}
    \renewcommand*{\arraystretch}{1.5}
    \begin{tabular}{| p{\colAwidth} | p{\colBwidth}|}
        \hline
        \rowcolor[gray]{0.9}
        Number& DD\refstepcounter{datadefnum}\thedatadefnum 
        \label{DD_interval}\\
        \hline
        Label& \bf Interval\\
        \hline
        Symbol &$t_0, t_N$\\
        \hline
        Units& N/A\\
        \hline
        %  SI Units & \si{\watt\per\square\metre}\\
        %  \hline
        Equation& N/A\\%all $ t \in [t_0, t_N]$\\
        \hline
        Description & 
        $[t_0,t_N] : \mathbb{R} \times \mathbb{R}$ is the interval for which to 
        solve the ODE (see \ddref{DD_ODE}). $t_0$ 
        represents the beginning of the interval, $t_N$ the end.
        \\
        \hline
  Sources & \cite{corless_graduate_2013} \\
  \hline
  Ref.\ By & \iref{IM_RK4}, \iref{IM_RK2}, \aref{A_ODEnonstiffcontinuous}, 
  \aref{A_interval}\\
        \hline
    \end{tabular}
\end{minipage}\\

~\newline

\noindent
\begin{minipage}{\textwidth}
    \renewcommand*{\arraystretch}{1.5}
    \begin{tabular}{| p{\colAwidth} | p{\colBwidth}|}
        \hline
        \rowcolor[gray]{0.9}
        Number& DD\refstepcounter{datadefnum}\thedatadefnum 
        \label{DD_initialvalues}\\
        \hline
        Label& \bf Initial values\\
        \hline
        Symbol & ${\bf x}_0$\\
        \hline
        Units& N/A\\
        \hline
        %  SI Units & \si{\watt\per\square\metre}\\
        %  \hline
        Equation& ${\bf x}(t_0) = {\bf x}_0$\\
        \hline
        Description & 
        ${\bf x}_0 : \mathbb{C}^n$ are initial values for solving ODE (see 
        \ddref{DD_ODE}).
        \\
        \hline
  Sources & \cite{corless_graduate_2013} \\
  \hline
  Ref.\ By & \ddref{DD_stepsize}, \iref{IM_RK4}, \iref{IM_RK2}, \tref{T_IVP}, 
  \aref{A_initialvalues}\\
        \hline
    \end{tabular}
\end{minipage}\\

~\newline

\noindent
\begin{minipage}{\textwidth}
    \renewcommand*{\arraystretch}{1.5}
    \begin{tabular}{| p{\colAwidth} | p{\colBwidth}|}
        \hline
        \rowcolor[gray]{0.9}
        Number& DD\refstepcounter{datadefnum}\thedatadefnum 
        \label{DD_stepsize}\\
        \hline
        Label& \bf Step size\\
        \hline
        Symbol & $h$ \\
        \hline
        Units& N/A \\
        \hline
        %  SI Units & \si{\watt\per\square\metre}\\
        %  \hline
        Equation& $t_{k+1} - t_k = h$\\
        \hline
        Description & 
        $h : \mathbb{R}$ is the distance between the points within interval 
        $[t_0,t_N]$ for which to find approximations.  %using RK methods (as 
        %described in 
        %\iref{IM_RK4} and \iref{IM_RK2}).
        \\
        \hline
  Sources & \cite{corless_graduate_2013} \\
  \hline
  Ref.\ By & \iref{IM_RK4}, \iref{IM_RK2}\\
        \hline
    \end{tabular}
\end{minipage}\\

\subsection{Goal Statements}

\noindent Given an RK method specification, the specified ODE (\ddref{DD_ODE}; 
non-stiff and continuous, as 
per 
\aref{A_ODEnonstiffcontinuous}), an interval (\ddref{DD_interval}) and the 
desired step 
size (\ddref{DD_stepsize}), as well as 
an initial condition (\ddref{DD_initialvalues}), the goal statement is:

\begin{itemize}

\item[GS\refstepcounter{goalnum}\thegoalnum \label{G_meaningfulLabel}:] Given 
Calculate solutions for and generate (in a 
type-safe manner) a 
function that uses these solutions to solve for specific points (multiples of 
the step size) within the provided 
interval.
Type-safe code generation means that the code snippets or fragments will have 
types, rather than being strings for example. If done properly, this allows 
only valid code to be generated.

In phase 2 (starting in January 2018), spline interpolation (between the points 
that fall on multiples of the step size) will be added. This 
will allow the generated function to solve for any point within the interval, 
rather than only specific ones.
\end{itemize}

\wss{Why are you mentioning a spline?  In addition to solving the ode at the
  specified points, are you interpolating between the points using a spline?  If
  so, you should add information defining a spline, and you'll need to be more
  specific on what spline you are using in terms of order and continuity
  conditions at the knots.}
\als{Unfortunately, it seems I was too ambitious. Splines have been moved to 
phase 2 that starts in January.}

\subsection{Theoretical Models} \label{sec_theoretical}

This section focuses on the problem that \famname{} is designed to solve.

~\newline

\noindent
\begin{minipage}{\textwidth}
\renewcommand*{\arraystretch}{1.5}
\begin{tabular}{| p{\colAwidth} | p{\colBwidth}|}
  \hline
  \rowcolor[gray]{0.9}
  Number& T\refstepcounter{theorynum}\thetheorynum \label{T_IVP}\\
  \hline
  Label&\bf Initial value problem (IVP)\\
  \hline
  Input &
  %${\bf x'} = {\bf f} (t,{\bf x} (t))$, \quad 
  %$[t_0 .. t_N]$, \quad 
  %${\bf x}(t_0) = {\bf x}_0$ \quad
  %where 
    \begin{itemize}
      %\item ${\bf x} : \mathbb{R} \rightarrow \mathbb{C}^n$ is the 
      %vector solution as a function of time
      \item ${\bf f} : 
      \mathbb{R} \times \mathbb{C}^n \rightarrow \mathbb{C}^n$ is the function 
      describing the vector field (see \ddref{DD_ODE})
      \item $t_0, t_N : \mathbb{R} \times \mathbb{R}$ the bounds of the 
      interval within which to solve (see \ddref{DD_interval})
      \item ${\bf x}_0 : 
      \mathbb{C}^n$ is the initial condition (see \ddref{DD_initialvalues})
      \item $h : \mathbb{R}$ the step size (see \ddref{DD_stepsize})
    \end{itemize} 
%   Given an initial value ${\bf x}_0$, the goal is to find approximations on a 
%   given interval.
    \\
  \hline
  Output & 
  \begin{itemize}
    \item anonymous function : $\mathbb{R} \rightarrow 
    \mathbb{R}^n$
  \end{itemize}
  This function takes in 
  %an index (to select a specific element of the output vector) and 
  a value $t$ for which to return an approximation of the given ODE at that 
  point. \\
  \hline
  Description & 
                The inputs of an initial value problem are given above. 
                The issue is that many IVPs are difficult to solve manually (or 
                programmatically) and the correct solutions are often unknown. 
                Numerical methods such as RK methods are close enough to be 
                used in most 
                applications.\\
  \hline
  Source &
           \cite{corless_graduate_2013} p. 510, 513\\
  % The above web link should be replaced with a proper citation to a publication
  \hline
  Ref.\ By & \iref{IM_RK4}, \iref{IM_RK2} \\
  \hline
\end{tabular}
\end{minipage}\\

\wss{It would be better if you used the instance model template for your
  theoretical model.  In particular, a clear separation between the inputs and
  the ouputs would be nice.}
\als{I separated the Equation row into Input and Output.}
\section{Variabilities}

Assumptions on all variations of the program family are covered first.
Variability in RK method in the \famname{} program family are represented by
instance models in \nameref{sec_Calculation}. Every unique input will result in
slightly different code being generated, which could in turn produce different
output compared to other family members. This allows for an infinite program
family depending on one's definition of program family\footnote{Currently
  Dr. Smith and Dr. Carette agree they should discuss this definition to compare
  their overlapping but slightly differing viewpoints. \wss{By default, LaTeX
    puts two spaces after a period, since it assumes that the period ends a
    sentence.  To override this, use x.\ y or x.~y.  The first puts one space,
    the second option also does one space, with the added constraint that a line
    break cannot happen at the space.}}. After instance models, the lack of
variability in output is covered.

\subsection{Assumptions}\label{ssec:Assumptions}

\begin{itemize}

\item[A\refstepcounter{assumpnum}\theassumpnum \label{A_RKorder}:] Only the 
second (\iref{IM_RK4}) and fourth (\iref{IM_RK2}) order RK methods will be 
used. 
\wss{This is a
  variability, but it isn't an assumption.} \als{I're changed this to reflect 
  which instance models are available as per Geneva's suggestion.}

\item[A\refstepcounter{assumpnum}\theassumpnum 
\label{A_ODEnonstiffcontinuous}:] The ODE (\ddref{DD_ODE}) provided will be 
continuous (particularly on the interval of computation (\ddref{DD_interval})) 
and 
non-stiff (otherwise the user accepts that the results will be inaccurate).

%\item[A\refstepcounter{assumpnum}\theassumpnum \label{A_initialvaluesata}:] 
%The 
%given initial values (\ddref{DD_initialvalues}) are for the beginning of the 
%interval (\ddref{DD_interval}), represented by $t_0$. 
\wss{The assumptions don't
  need to repeat things you have alread said.} \als{Removed this assumption.}

\item[A\refstepcounter{assumpnum}\theassumpnum \label{A_initialvalues}:] 
Initial value (\ddref{DD_initialvalues}) vector size is expected to match the 
ones produced by the ODE 
(\ddref{DD_ODE}).

\item[A\refstepcounter{assumpnum}\theassumpnum \label{A_interval}:] The 
interval's bounds (\ddref{DD_interval}) satisfy $t_0 < t_N$ (both real 
numbers) and $t_N - t_0 = n 
\times h$. This will make adding interpolation in phase 2 easier.
  \wss{In general with floating point arithmetic this won't necessarily
  be true.  In practice if $h$ is added to the time for each time step, you
  won't end exactly at $t_N$.  Do you really need this assumption to be true?} 
  \als{No, but it is useful for when interpolation is added.}

%\item[A\refstepcounter{assumpnum}\theassumpnum \label{A_stepsize}:] The step 
%size (\ddref{DD_stepsize}) shall be a positive real number. 
  \wss{You can give
  this through the type of $h$; you don't need an assumption for this 
  information.} \als{I've removed this assumption as well.}
\end{itemize}

\wss{It is okay to have a short list of assumptions.  It feels like you might
  have been trying to ``pad'' the list.  What about an assumption that $f$ is
  continuous on the interval of computation?}
\als{I've added this to \aref{A_ODEnonstiffcontinuous}}

\subsection{Calculation} \label{sec_Calculation}

The family can utilize either of the instance models listed below for any of 
its members.
As stated in \iref{IM_One-step_IVP_solver}, the functions $\phi$ and ${\bf f} 
(t_k, {\bf x}_k)$ constitute two variabilities, as different definitions of 
either of these will yield different program family members.

\wss{To make this clear that you have a family of methods, I would introduce an
  instance model for one-step initial value problem solvers.  The equation would
  be $${\bf x}_{k+1} = {\bf x}_k + h \phi(t_k, {\bf x}_k, h_k)$$  The only
  difference for your family members is how $\phi$ is defined.  This approach
  would let you easily add new family members in the specification, simply by
  giving new values for $\phi$.}
\als{I've done this; I hope I'm still using the template constructs correctly. 
I'm considering the possibility of moving them to the theoretical models 
section, but not sure how I would convey that these constitute variabilities by 
doing so, except for adding text in the description of 
\iref{IM_One-step_IVP_solver}. 
I removed the $h_k$ parameter, as I'm not considering variable step size.}

\noindent
\begin{minipage}{\textwidth}
  \renewcommand*{\arraystretch}{1.5}
  \begin{tabular}{| p{\colAwidth} | p{\colBwidth}|}
    \hline
    \rowcolor[gray]{0.9}
    Number & IM\refstepcounter{instnum}\theinstnum 
    \label{IM_One-step_IVP_solver}\\
    \hline
    Label &\bf One-step IVP solver\\
    \hline
    Equations &  
    \begin{itemize}
      \item ${\bf x}_{k+1} = {\bf x}_k + h \phi(t_k, {\bf x}_k)$
    \end{itemize}\\
    \hline
    Description & 
    The above equation provides the skeleton can be used to calculate (an 
    approximation of) a 
    new point given the previous or starting point, given some definition of 
    the $\phi$ function, two of which are given below (\iref{IM_RK2}, 
    \iref{IM_RK4}). ${\bf f} (t_k, {\bf x}_k)$ is another variability, since 
    each definition of ${\bf f} (t_k, {\bf x}_k)$ will yield another program 
    family member.
    In addition to the ODE (\ddref{DD_ODE}) this 
    also requires an interval (\ddref{DD_interval}), step size 
    (\ddref{DD_stepsize}) and initial condition 
    (\ddref{DD_initialvalues}).\\
    \hline
    Source &
    Dr.~W.~S.~Smith \\
    % The above web link should be replaced with a proper citation to a 
    %publication
    \hline
    Ref.\ By & \iref{IM_RK2}, \iref{IM_RK4}\\
    \hline
  \end{tabular}
\end{minipage}\\

\noindent
\begin{minipage}{\textwidth}
  \renewcommand*{\arraystretch}{1.5}
  \begin{tabular}{| p{\colAwidth} | p{\colBwidth}|}
    \hline
    \rowcolor[gray]{0.9}
    Number & IM\refstepcounter{instnum}\theinstnum \label{IM_RK2}\\
    \hline
    Label &\bf Second-order Runge-Kutta method (Improved Euler method)\\
    \hline
    Equations &  
    \begin{itemize}
      \item $t_{k+1} = t_{k} + h$
      \item ${\bf Y}_1 = {\bf x}_k + h {\bf f} (t_k, {\bf x}_k)$ %new 
      %intermediate point at x_{k+1} by extrapolating slope at x_k
      \item $\phi(t_k, {\bf x}_k) = \Big( \dfrac{1}{2} {\bf f} 
      (t_k,{\bf x}_k) + \dfrac{1}{2} {\bf f} (t_{k+1}, {\bf Y}_1) \Big)$ 
      %new point by averaging slope at current point and slope at new 
      %point for extrapolating from current point
    \end{itemize}\\
    \hline
    Description & 
    The above equations lead to the definition of a function $\phi$, which can 
    be used to calculate (an approximation of) a 
    new point given the previous point in conjunction with 
    \iref{IM_One-step_IVP_solver}. In addition to the ODE 
    (\ddref{DD_ODE}) this 
    also requires an interval (\ddref{DD_interval}), step size 
    (\ddref{DD_stepsize}) and initial condition 
    (\ddref{DD_initialvalues}).\\
    \hline
    Source &
    \cite{corless_graduate_2013} p. 616 \\
    % The above web link should be replaced with a proper citation to a 
    %publication
    \hline
    Ref.\ By & \aref{A_ODEnonstiffcontinuous}\\
    \hline
  \end{tabular}
\end{minipage}\\

~\newline

~\newline

\noindent
\begin{minipage}{\textwidth}
    \renewcommand*{\arraystretch}{1.5}
    \begin{tabular}{| p{\colAwidth} | p{\colBwidth}|}
        \hline
        \rowcolor[gray]{0.9}
        Number& IM\refstepcounter{instnum}\theinstnum \label{IM_RK4}\\
        \hline
        Label&\bf Fourth order Runge-Kutta method (RK4)\\
        \hline
        Equation&  
        \begin{itemize}
            \item $t_{k+1} = t_{k} + h$
            \item ${\bf k}_1 = {\bf f} (t_k, {\bf x}_k)$ 
            \item ${\bf k}_2 = {\bf f} (t_k + \dfrac{h}{2}, {\bf x}_k + 
            \dfrac{h}{2} {\bf k}_1)$
            \item ${\bf k}_3 = {\bf f} (t_k + \dfrac{h}{2}, {\bf x}_k + 
            \dfrac{h}{2} {\bf k}_2)$
            \item ${\bf k}_4 = {\bf f} (t_k + h, {\bf x}_k + 
            h {\bf k}_3)$
            \item $\phi(t_k,{\bf x}_k) = \dfrac{1}{6} ( {\bf k}_1 + 2 
            {\bf k}_2 + 2 {\bf k}_3 + {\bf k}_4)$ new point created when 
            extrapolating from point ${\bf x}_k$ using a weighted average of 
            the previously calculated slopes
            
        \end{itemize}\\
        \hline
        Description & 
        The above equations lead to the definition of a function $\phi$, which 
        can be used to calculate (an approximation of) a 
        new point given the previous point in conjunction with 
        \iref{IM_One-step_IVP_solver}.
%        Note that the above is a specific instance (for simplicity) of the 
%        general method (namely, the first step). 
        \wss{Why specify the first step?  Why not specify the $k$th
          step?}
        \als{I copied a textbook example and forgot to generalize it, I've 
        updated it to reflect the $k$th step.}
        This method starts by calculating the slope at $x_k$, represented by 
        ${\bf k}_1$. Extrapolating this slope ${\bf k}_1$ from point ${\bf 
          x}_k$, it calculates a slope for the point halfway between $t_k$ 
        and $t_{k+1}$. Subsequently, it calculates a better slope for the point 
        halfway between $t_k$ and $t_{k+1}$ when extrapolating slope ${\bf 
        k}_2$ from point ${\bf x}_k$. Finally, it calculates the slope of the 
        point at $t_{k+1}$ by extrapolating slope ${\bf k}_3$ from point ${\bf 
        x}_k$.
%        Calculating a new point is necessary to create a spline to solve IVPs 
%(\tref{T_IVP}).
        In addition to the ODE (\ddref{DD_ODE}), this 
        also requires an interval (\ddref{DD_interval}), step size 
        (\ddref{DD_stepsize}) and initial condition 
        (\ddref{DD_initialvalues}).\\
        \hline
        Source &
        \cite{corless_graduate_2013} p. 618\\
        % The above web link should be replaced with a proper citation to a 
        %publication
        \hline
        Ref.\ By & \aref{A_ODEnonstiffcontinuous}\\
        \hline
    \end{tabular}
\end{minipage}\\

\wss{Again, I'm not sure how splines fit into this?}
\als{Mostly via my optimism at the time, so I've removed the reference to it.}

~\newline


\wss{Why isn't the function $f$ listed as a variability?  You get a different
  program depending on the value of $f$.}
\als{I've added this as part of the description of 
\ref{IM_One-step_IVP_solver}.}
\subsection{Output} \label{sec_Output}
Every program family member consists of a function that will output a numerical 
approximation of an ODE at a particular point on the given interval (which 
should be a multiple of the step size added to the lower bound of the interval).

The accuracy of the generated function may vary depending the family member 
that produced it and the ODE that was approximated.

\section{Traceability Matrices and Graphs}
The traceability matrices below show the relationships between the various 
concepts defined in this document. Rows may be affected by changes in the items 
the columns consist of (especially in the context of assumptions.)
\begin{table}[htb]
  \centering
  \begin{tabular}{lcccccccc}
  \toprule
    & \ddref{DD_ODE} & \ddref{DD_interval} & \ddref{DD_initialvalues} & 
    \ddref{DD_stepsize} & \tref{T_IVP} & \iref{IM_One-step_IVP_solver} & 
    \iref{IM_RK2} & \iref{IM_RK4} \\ 
    \midrule
\ddref{DD_ODE} &  &  &  &  & \checkmark &  &  &  \\ 
\ddref{DD_interval}    &  &  &  &  & \checkmark &  &  &  \\ 
\ddref{DD_initialvalues}    & \checkmark &  &  &  & \checkmark &  &  &  \\ 
\ddref{DD_stepsize}    &  & \checkmark &  &  & \checkmark &  &  &  \\ 
\tref{T_IVP}    &  &  &  &  &  &  &  &  \\ 
\iref{IM_One-step_IVP_solver}    & \checkmark & \checkmark & \checkmark & 
\checkmark & 
\checkmark &  &  &  \\ 
\iref{IM_RK2}    & \checkmark & \checkmark & \checkmark & \checkmark & 
\checkmark & \checkmark &  &  \\ 
\iref{IM_RK4}    & \checkmark & \checkmark & \checkmark & \checkmark & 
\checkmark & \checkmark &  &  
\\ 
\bottomrule
\end{tabular}
\caption{Traceability matrix between instance models, data definitions and 
theory}
\end{table}

\begin{table}[htb]
  \centering
  \begin{tabular}{lcccccc}
  \toprule 
  & \aref{A_RKorder} & \aref{A_ODEnonstiffcontinuous} & \aref{A_initialvalues} 
  & \aref{A_interval} \\ 
  \midrule 
\ddref{DD_ODE} &  & \checkmark & \checkmark &  \\ 
\ddref{DD_interval} &  & \checkmark &  & \checkmark \\ 
\ddref{DD_initialvalues} &  &  & \checkmark &  \\ 
\ddref{DD_stepsize} &  &  &  &  \\ 
\tref{T_IVP} &  &  &  &  &  \\ 
\iref{IM_One-step_IVP_solver} &  &  &  &  \\ 
\iref{IM_RK2} & \checkmark & \checkmark &  &  \\ 
\iref{IM_RK4} & \checkmark & \checkmark &  &  \\ 
  \bottomrule 
\end{tabular}
\caption{Traceability matrix for assumptions}
\end{table}
\clearpage
\newpage

\bibliographystyle {plainnat}
\bibliography {../References,../ReferencesA}

%\newpage
%
%\section{Appendix}
%
%\wss{Your report may require an appendix.  For instance, this is a good point 
%to
%show the values of the symbolic parameters introduced in the report.}
%
%\subsection{Symbolic Parameters}
%
%\wss{The definition of the requirements will likely call for 
%SYMBOLIC\_CONSTANTS.
%Their values are defined in this section for easy maintenance.}

\end{document}