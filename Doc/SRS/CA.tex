\documentclass[12pt]{article}

\usepackage{amsmath, mathtools}
\usepackage{amsfonts}
\usepackage{amssymb}
\usepackage{graphicx}
\usepackage{colortbl}
\usepackage{xr}
\usepackage{hyperref}
\usepackage{longtable}
\usepackage{xfrac}
\usepackage{tabularx}
\usepackage{float}
\usepackage{siunitx}
\usepackage{booktabs}
\usepackage{caption}
\usepackage{pdflscape}
\usepackage{afterpage}

\usepackage[round]{natbib}

%\usepackage{refcheck}

\hypersetup{
    bookmarks=true,         % show bookmarks bar?
      colorlinks=true,       % false: boxed links; true: colored links
    linkcolor=red,          % color of internal links (change box color with linkbordercolor)
    citecolor=green,        % color of links to bibliography
    filecolor=magenta,      % color of file links
    urlcolor=cyan           % color of external links
}

%% Comments

\usepackage{color}

\newif\ifcomments\commentstrue

\ifcomments
\newcommand{\authornote}[3]{\textcolor{#1}{[#3 ---#2]}}
\newcommand{\todo}[1]{\textcolor{red}{[TODO: #1]}}
\else
\newcommand{\authornote}[3]{}
\newcommand{\todo}[1]{}
\fi

\newcommand{\wss}[1]{\authornote{blue}{SS}{#1}}
\newcommand{\als}[1]{\authornote{magenta}{AS}{#1}}


% For easy change of table widths
\newcommand{\colZwidth}{1.0\textwidth}
\newcommand{\colAwidth}{0.13\textwidth}
\newcommand{\colBwidth}{0.82\textwidth}
\newcommand{\colCwidth}{0.1\textwidth}
\newcommand{\colDwidth}{0.05\textwidth}
\newcommand{\colEwidth}{0.8\textwidth}
\newcommand{\colFwidth}{0.17\textwidth}
\newcommand{\colGwidth}{0.5\textwidth}
\newcommand{\colHwidth}{0.28\textwidth}

% Used so that cross-references have a meaningful prefix
\newcounter{defnum} %Definition Number
\newcommand{\dthedefnum}{GD\thedefnum}
\newcommand{\dref}[1]{GD\ref{#1}}
\newcounter{datadefnum} %Datadefinition Number
\newcommand{\ddthedatadefnum}{DD\thedatadefnum}
\newcommand{\ddref}[1]{DD\ref{#1}}
\newcounter{theorynum} %Theory Number
\newcommand{\tthetheorynum}{T\thetheorynum}
\newcommand{\tref}[1]{T\ref{#1}}
\newcounter{tablenum} %Table Number
\newcommand{\tbthetablenum}{T\thetablenum}
\newcommand{\tbref}[1]{TB\ref{#1}}
\newcounter{assumpnum} %Assumption Number
\newcommand{\atheassumpnum}{P\theassumpnum}
\newcommand{\aref}[1]{A\ref{#1}}
\newcounter{goalnum} %Goal Number
\newcommand{\gthegoalnum}{P\thegoalnum}
\newcommand{\gsref}[1]{GS\ref{#1}}
\newcounter{instnum} %Instance Number
\newcommand{\itheinstnum}{IM\theinstnum}
\newcommand{\iref}[1]{IM\ref{#1}}
\newcounter{reqnum} %Requirement Number
\newcommand{\rthereqnum}{P\thereqnum}
\newcommand{\rref}[1]{R\ref{#1}}
\newcounter{lcnum} %Likely change number
\newcommand{\lthelcnum}{LC\thelcnum}
\newcommand{\lcref}[1]{LC\ref{#1}}

\newcommand{\famname}{rkf45.ml} % PUT YOUR PROGRAM NAME HERE

\usepackage{fullpage}
\usepackage{tikz}
\usetikzlibrary{arrows}

\begin{document}

\title{RKF45 Generator} 
\author{Alexander Schaap}
\date{\today}

\maketitle

~\newpage

\pagenumbering{roman}

\section{Revision History}

\begin{tabularx}{\textwidth}{p{3cm}p{2cm}X}
\toprule {\bf Date} & {\bf Version} & {\bf Notes}\\
\midrule
\today & 1.0 & Initial version\\
\bottomrule
\end{tabularx}

~\newpage
	
\section{Reference Material}

This section records information for easy reference.

\subsection{Table of Units}

Since the application of this software is dependent on the program using the 
functions provided by this program family, no physical units are used 
throughout this document.

\subsection{Table of Symbols}

The table that follows summarizes the symbols used in this document along with
their units.  The choice of symbols was made to be consistent with the heat
transfer literature and with existing documentation for solar water heating
systems.  The symbols are listed in alphabetical order.

\renewcommand{\arraystretch}{1.2}
%\noindent \begin{tabularx}{1.0\textwidth}{l l X}
\noindent \begin{longtable*}{l l p{12cm}} \toprule
\textbf{symbol} & \textbf{unit} & \textbf{description}\\
\midrule 
$A_C$ & \si[per-mode=symbol] {\square\metre} & coil surface area
\\
$A_\text{in}$ & \si[per-mode=symbol] {\square\metre} & surface area over 
which heat is transferred in
\\ 
\bottomrule
\end{longtable*}
\wss{Use your problems actual symbols.  The si package is a good idea to use for
  units.}

\subsection{Abbreviations and Acronyms}

\renewcommand{\arraystretch}{1.2}
\begin{tabular}{l l} 
  \toprule		
  \textbf{symbol} & \textbf{description}\\
  \midrule 
  A & Assumption\\
  DD & Data Definition\\
  GD & General Definition\\
  GS & Goal Statement\\
  IM & Instance Model\\
  LC & Likely Change\\
  ODE & Ordinary Differential Equation\\
  PS & Physical System Description\\
  R & Requirement\\
  RK & Runge-Kutta\\
  RK2 & Second order Runge-Kutta method\\
  RK4 & Fourth order Runge-Kutta method\\
%  RKF45 & Runge-Kutta-Fehlberg method\\
  SRS & Software Requirements Specification\\
  \famname{} & Family of programs based on the RK4 / RKF45 method(s)\\
  T & Theoretical Model\\
  \bottomrule
\end{tabular}\\

\wss{Add any other abbreviations or acronyms that you add}

\newpage

\tableofcontents

~\newpage

\pagenumbering{arabic}

\section{Introduction}
%modelled after 3dfim+
This document provides an overview of the commonality analysis (CA) for the 
\famname{} program family. Members of program family are produced by a code 
generator. Generated members provide numerical approximations for given 
ordinary differential equations (ODEs) using Runge-Kutta (RK) methods. Most of 
the calculations happens during code generation, producing a different family 
member for each given combination of RK method, ODE, interval, step size, and 
initial values. 
%The family member presents the results in the form of a function that can be 
%called. 
The current section describes the purpose of this document, the scope of this 
family, the organization of the remainder of the document and the 
characteristics of the intended reader.

\subsection{Purpose of Document}
%modelled after 3dfim+
The main purpose of this document is to provide sufficient information to 
understand what \famname{} is expected to do. The goals and theoretical models 
used in the 
\famname{} generator are provided, as are assumptions and unambiguous 
definitions.

\subsection{Scope of the Family}
%loosely modelled after 3dfim+
The responsibility of the family is solving the provided ODEs for given 
intervals, initial values and step sizes using the specified RK method. The 
user will have to ensure the RK method specified is appropriate for their 
particular problem(s), and they will have to ensure appropriate and correct 
input is provided (see \ref{ssec:Assumptions} for some starting points). 
Additionally, the user will also have to process the output appropriately.

\subsection{Characteristics of Intended Reader} 
The reader is expected to have some undergraduate STEM and functional 
programming background. Ideally, they have been exposed to some calculus and 
programming courses.

\subsection{Organization of Document}
%inspired by FamilyOfMaterialModels
The organization of this document follows the template for a commonality 
analysis (CA) provided by Dr. W. S. Smith, which is in turn adapted from 
\cite{Lai2004},\cite{SmithAndLai2005}, \cite{Smith2006} and 
\cite{SmithEtAl2007}.

The structure of this document is essentially top-down, with detail added as 
one proceeds through it.

\section{General System Description}

This section identifies the interfaces between the system and its environment,
describes the potential user characteristics and lists the potential system
constraints.

\subsection{Potential System Contexts}

\autoref{fig:systemcontext} shows the system context. A circle represents than 
external entity outside the software, which is the ``user'', or parent program 
making use of the generator as a library. A rectangle is used to represent the 
generator as well as the \famname{} family members it can generate. 

Ideally, 
the parent program knows the ODE(s) and other inputs at compile time, and
uses each generated family member many times at run-time to maximize the 
benefit of code generation. 
Alternatively, the ODE(s) or other inputs are 
not known at compile time; code generation can still happen at run-time, but 
hopefully the parent program can offset the incurred cost of this on-the-fly 
generation by running the optimized generated code enough times.

\begin{figure}[ht]
    \centering
\begin{tikzpicture} [
    auto,
    c/.style = {circle, draw, text centered},
    b/.style = {rectangle, draw, text centered, rounded corners, minimum 
    height=2em},
    arrow/.style = {draw, ->},
    node distance = 5cm
    ]
    \node[c](user) {Parent program};
    \node[b, right of=user](RKgen) {\famname{} Generator}; 
    \node[b, below of=RKgen](gen) {Generated code};
    
    \draw[arrow] (user) -- node {Inputs} (RKgen);
    \draw[arrow] (RKgen) -- node {Produces} (gen);
    \draw[arrow] (user) -- node {Calls} (gen);
    \draw[arrow] (gen) -| node {Results} (user);
    
\end{tikzpicture}
\caption{System Context}
\label{fig:systemcontext}
\end{figure}

\begin{itemize}
\item User Responsibilities:
\begin{itemize}
\item Provide appropriate input to the system
\item Ensure assumptions on the input are met (see \autoref{ssec:Assumptions})
\item Correctly process the resulting output
\end{itemize}
\item \famname{} Responsibilities:
\begin{itemize}
\item Calculate the correct output for the given input
\end{itemize}
\end{itemize}

\subsection{Potential User Characteristics} \label{SecUserCharacteristics}

The most common user of \famname{} will be other programs. However, one or more 
programmers are needed to write the code that calls the functions provided by 
this family.
These programmers therefore should have an understanding of undergraduate 
calculus and OCaml programming. (Knowledge of MetaOCaml should not be necessary 
if the program family is able to hide its usage thereof sufficiently).
                                                                                
                          
\subsection{Potential System Constraints}

The responsibility of generating family members in a type-safe way restricts 
the system to the MetaOCaml extension of the OCaml programming language.

\section{Commonalities}

This section contains the commonalities between the \famname{} family members. 
This section starts with succinct background information about the problem, 
followed by terminology and definitions, data definitions, goal statements and 
ending with theoretical models. 

\subsection{Background Overview} \label{Sec_Background}

There are various numerical methods for approximating ordinary differential 
equation (ODE) given initial values. This problem is called an initial value 
problem (IVP, see T\tref{T_IVP}). A well-known way to solve these problems is 
through Runge-Kutta methods.% (which include Euler's method).

\subsection{Terminology and  Definitions}

This subsection provides a list of terms that are used in the subsequent
sections and their meaning, with the purpose of reducing ambiguity and making it
easier to correctly understand the requirements:

\begin{itemize}

\item Ordinary differential equation (ODE): an equation that involves some 
ordinary (rather than partial) derivatives of a function. The goal is to find 
the function that satisfies the equation, and this is not always easily 
achievable through integration. (See D\dref{DD_ODE}.)

\item Continuous function: a function that evaluates to a value that is not 
infinity or undefined for any input.

\item Stiff ODE: an ODE for which certain numerical methods are numerically 
unstable unless the step size is extremely small. There is no exact definition.

\item Initial values (initial condition): a starting point from which the rest 
of the approximation 
can be calculated. (See D\dref{DD_initialvalues}.)

\item Interval: interval inside which the ODE needs to be approximated. (See 
\dref{DD_interval}.)

\item Step size: the distance between the points that need to be calculated 
using an RK method, for example. 

\item Initial value problem (IVP): an ODE along with initial values for which a 
numerical approximation must be found.

\item Runge-Kutta methods: a set of methods that approximate ODEs in IVPs for a 
given interval and step size.

\item Compile-time: the time during which a program is compiled.

\item Run-time: the time during which a program is run.

\item Metaprogramming: programming software that takes in and/or produces 
programs.

\item Code generation: producing code programmatically.

\item Multi-stage programming (MSP): a form of metaprogramming where the 
execution of specific code fragments is delayed; these fragments are then 
combined into larger fragments and ultimately executed.

\item MetaOCaml: an extension to the OCaml programming language which adds 
staging annotations that enable MSP.

\end{itemize}

\subsection{Data Definitions} \label{sec_datadef}

This section collects and defines all the data needed to build the instance
models. The dimension of each quantity is also given.
~\newline

\noindent
\begin{minipage}{\textwidth}
\renewcommand*{\arraystretch}{1.5}
\begin{tabular}{| p{\colAwidth} | p{\colBwidth}|}
\hline
\rowcolor[gray]{0.9}
Number& DD\refstepcounter{datadefnum}\thedatadefnum \label{DD_ODE}\\
\hline
Label& \bf Ordinary Differential Equation (ODE)\\
\hline
Symbol &${\bf f}$\\
\hline
 Units& N/A\\
 \hline
%  SI Units & \si{\watt\per\square\metre}\\
%  \hline
  Equation&${\bf x'} = {\bf f} (t,{\bf x} (t))$\\
  \hline
  Description & 
                 ${\bf f} : \mathbb{R} \times \mathbb{C}^n \rightarrow 
                 \mathbb{C}^n$ is the equation for which we ultimately want to 
                 find 
                 a numerical approximation.
  \\
  \hline
  Sources & \cite{corless_graduate_2013} \\
  \hline
  Ref.\ By & \iref{T_RK4}, \iref{IM_RK2}\\
  \hline
\end{tabular}
\end{minipage}\\

~\newline

\noindent
\begin{minipage}{\textwidth}
    \renewcommand*{\arraystretch}{1.5}
    \begin{tabular}{| p{\colAwidth} | p{\colBwidth}|}
        \hline
        \rowcolor[gray]{0.9}
        Number& DD\refstepcounter{datadefnum}\thedatadefnum 
        \label{DD_initialvalues}\\
        \hline
        Label& \bf Initial values\\
        \hline
        Symbol &$a, b$\\
        \hline
        Units& N/A\\
        \hline
        %  SI Units & \si{\watt\per\square\metre}\\
        %  \hline
        Equation&\\
        \hline
        Description & 
        $[a,b] : \mathbb{R} \times \mathbb{R}$ is the interval for which to 
        solve the ODE (see \ddref{DD_ODE}). $a$ 
        represents the beginning of the interval, $b$ the end.
        \\
        \hline
  Sources & \cite{corless_graduate_2013} \\
  \hline
  Ref.\ By & \iref{T_RK4}, \iref{IM_RK2}\\
        \hline
    \end{tabular}
\end{minipage}\\

~\newline

\noindent
\begin{minipage}{\textwidth}
    \renewcommand*{\arraystretch}{1.5}
    \begin{tabular}{| p{\colAwidth} | p{\colBwidth}|}
        \hline
        \rowcolor[gray]{0.9}
        Number& DD\refstepcounter{datadefnum}\thedatadefnum 
        \label{DD_interval}\\
        \hline
        Label& \bf Interval\\
        \hline
        Symbol & ${\bf x}_0$\\
        \hline
        Units& N/A\\
        \hline
        %  SI Units & \si{\watt\per\square\metre}\\
        %  \hline
        Equation& ${\bf x}(t_0) = {\bf x}_0$\\
        \hline
        Description & 
        ${\bf x} : \mathbb{C}^n$ are initial values for solving ODE (see 
        \ddref{DD_ODE}).
        \\
        \hline
  Sources & \cite{corless_graduate_2013} \\
  \hline
  Ref.\ By & \iref{T_RK4}, \iref{IM_RK2}\\
        \hline
    \end{tabular}
\end{minipage}\\

~\newline

\noindent
\begin{minipage}{\textwidth}
    \renewcommand*{\arraystretch}{1.5}
    \begin{tabular}{| p{\colAwidth} | p{\colBwidth}|}
        \hline
        \rowcolor[gray]{0.9}
        Number& DD\refstepcounter{datadefnum}\thedatadefnum 
        \label{DD_stepsize}\\
        \hline
        Label& \bf Step size\\
        \hline
        Symbol & $h$ \\
        \hline
        Units& N/A \\
        \hline
        %  SI Units & \si{\watt\per\square\metre}\\
        %  \hline
        Equation& N/A\\
        \hline
        Description & 
        $h : \mathbb{R}$ is the size of the steps between the points for which 
        to find approximations 
        using RK methods (as described in \iref{T_RK4} and \iref{IM_RK2}).
        \\
        \hline
  Sources & \cite{corless_graduate_2013} \\
  \hline
  Ref.\ By & \iref{T_RK4}, \iref{IM_RK2}\\
        \hline
    \end{tabular}
\end{minipage}\\

\subsection{Goal Statements}

\noindent Given the non-stiff continuous ODE, the goal statements are:

\begin{itemize}

\item[GS\refstepcounter{goalnum}\thegoalnum \label{G_meaningfulLabel}:] Given 
an RK method specification, an interval and the desired step size, as well as 
an initial value, calculate a spline and generate (in a type-safe manner) a 
function that uses this spline to solve for specific points within the provided 
interval.

\end{itemize}

\subsection{Theoretical Models} \label{sec_theoretical}

This section focuses on the general equations and laws that \famname{} is based
on.

~\newline

\noindent
\begin{minipage}{\textwidth}
\renewcommand*{\arraystretch}{1.5}
\begin{tabular}{| p{\colAwidth} | p{\colBwidth}|}
  \hline
  \rowcolor[gray]{0.9}
  Number& T\refstepcounter{theorynum}\thetheorynum \label{T_IVP}\\
  \hline
  Label&\bf Initial value problem (IVP)\\
  \hline
  Equation&  ${\bf x'} = {\bf f} (t,{\bf x} (t))$, \quad ${\bf 
  x}(t_0) = {\bf x}_0$ \quad
    where 
    \begin{itemize}
        \item ${\bf x} : \mathbb{R} \rightarrow \mathbb{C}^n$ is the 
    vector solution as a function of time
        \item ${\bf x}_0 \in 
        \mathbb{C}^n$ is the initial condition (see \ddref{DD_initialvalues})
        \item ${\bf f} : 
    \mathbb{R} \times \mathbb{C}^n \rightarrow \mathbb{C}^n$ is the function 
    describing the vector field (see \ddref{DD_ODE})
   \end{itemize} 
   Given an initial value ${\bf x}_0$, the goal is to find approximations on a 
   given interval $[t_a .. t_b]$.\\
  \hline
  Description & 
                The standard form of an initial value problem is given above. 
                The issue is that many IVPs are difficult to solve manually (or 
                programmatically) and the correct solutions are often unknown. 
                Numerical methods are close enough to be used in most 
                applications.\\
  \hline
  Source &
           \cite{corless_graduate_2013} p. 510, 513\\
  % The above web link should be replaced with a proper citation to a publication
  \hline
  Ref.\ By & \dref{ROCT}\\
  \hline
\end{tabular}
\end{minipage}\\


\section{Variabilities}
\subsection{Instance Models}
\noindent
\begin{minipage}{\textwidth}
    \renewcommand*{\arraystretch}{1.5}
    \begin{tabular}{| p{\colAwidth} | p{\colBwidth}|}
        \hline
        \rowcolor[gray]{0.9}
        Number& IM\refstepcounter{instnum}\theinstnum \label{T_RK4}\\
        \hline
        Label&\bf Fourth order Runge-Kutta method (RK4)\\
        \hline
        Equation&  
        \begin{itemize}
            \item $t_1 = t_0 + h$
            \item ${\bf k}_1 = {\bf f} (t_0, {\bf x}_0)$ slope at $x_0$
            \item ${\bf k}_2 = {\bf f} (t_0 + \dfrac{h}{2}, {\bf x}_0 + 
            \dfrac{h}{2} {\bf k}_1)$ slope of the point halfway between $t_0$ 
            and $t_1$ when extrapolating slope ${\bf k}_1$ from point ${\bf 
                x}_0$
            \item ${\bf k}_3 = {\bf f} (t_0 + \dfrac{h}{2}, {\bf x}_0 + 
            \dfrac{h}{2} {\bf k}_2)$ slope of the point halfway between $t_0$ 
            and $t_1$ when extrapolating slope ${\bf k}_2$ from point ${\bf 
                x}_0$
            \item ${\bf k}_4 = {\bf f} (t_0 + h, {\bf x}_0 + 
            h {\bf k}_3)$ slope of point at $t_1$ when extrapolating slope 
            ${\bf k}_3$ from point ${\bf x}_0$
            \item ${\bf x}_1 = {\bf x}_0 + \dfrac{h}{6} ( {\bf k}_1 + 2 {\bf 
                k}_2 + 2 {\bf k}_3 + {\bf k}_4)$ new point created when 
            extrapolating from point ${\bf x}_0$ using a weighted average of 
            the previously calculated slopes
            
        \end{itemize}\\
        \hline
        Description & 
        The above equations can be used to calculate (an approximation of) a 
        new point given the previous or starting point.\\
        \hline
        Source &
        \cite{corless_graduate_2013} p. 618\\
        % The above web link should be replaced with a proper citation to a 
        %publication
        \hline
        Ref.\ By & \dref{ROCT}\\
        \hline
    \end{tabular}
\end{minipage}\\

~\newline
\noindent
\begin{minipage}{\textwidth}
    \renewcommand*{\arraystretch}{1.5}
    \begin{tabular}{| p{\colAwidth} | p{\colBwidth}|}
        \hline
        \rowcolor[gray]{0.9}
        Number & IM\refstepcounter{instnum}\theinstnum \label{IM_RK2}\\
        \hline
        Label &\bf Second-order Runge-Kutta method (Improved Euler method)\\
        \hline
        Equations &  
        \begin{itemize}
            \item ${\bf Y}_1 = {\bf x}_k + h {\bf f} (t_k + {\bf x}_k)$
            \item ${\bf x}_{k+1} = {\bf x}_k + h \Big( \dfrac{1}{2} {\bf f} 
            (t_k,{\bf x}_k) + \dfrac{1}{2} {\bf f} (t_{k+1}, {\bf Y}_1) \Big)$
        \end{itemize}\\
        \hline
        Description & 
        The above equations can be used to calculate (an approximation of) a 
        new point given the previous or starting point.\\
        \hline
        Source &
        \cite{corless_graduate_2013} p. 616 \\
        % The above web link should be replaced with a proper citation to a 
        %publication
        \hline
        Ref.\ By & \dref{ROCT}\\
        \hline
    \end{tabular}
\end{minipage}\\

~\newline

\subsection{Assumptions}\label{ssec:Assumptions}

\begin{itemize}

\item[A\refstepcounter{assumpnum}\theassumpnum \label{A_meaningfulLabel}:]
  \wss{Short description of each assumption.  Each assumption
    should have a meaningful label.  Use cross-references to identify the
    appropriate traceability to T, GD, DD etc., using commands like dref, ddref 
    etc.}
\item[A\refstepcounter{assumpnum}\theassumpnum \label{A_initialvaluesata}:] The 
given initial values (\ddref{DD_initialvalues}) are for the beginning of the 
interval (\ddref{DD_interval}), represented by $a$.

\item[A\refstepcounter{assumpnum}\theassumpnum \label{A_initialvalues}:] 
Initial value vector size is expected to match the ones produced by the ODE 
(\ddref{DD_ODE}).

\item[A\refstepcounter{assumpnum}\theassumpnum \label{A_interval}:] The 
interval's bounds satisfy $a \leq b$.

\end{itemize}

\subsection{Calculation} \label{sec_Calculation}

\subsection{Output} \label{sec_Output}    

\section{Traceability Matrices and Graphs}

\wss{You will have to add tables.}

\newpage

\bibliographystyle {plainnat}
\bibliography {../References,../ReferencesA}

%\newpage
%
%\section{Appendix}
%
%\wss{Your report may require an appendix.  For instance, this is a good point 
%to
%show the values of the symbolic parameters introduced in the report.}
%
%\subsection{Symbolic Parameters}
%
%\wss{The definition of the requirements will likely call for 
%SYMBOLIC\_CONSTANTS.
%Their values are defined in this section for easy maintenance.}

\end{document}