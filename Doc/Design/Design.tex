\documentclass[12pt, titlepage]{article}

\usepackage{amsmath, mathtools}
\usepackage{amsfonts}
\usepackage{amssymb}
\usepackage{fullpage}
\usepackage[round]{natbib}
\usepackage{multirow}
\usepackage{booktabs}
\usepackage{tabularx}
\usepackage{graphicx}
\usepackage{float}
\usepackage{hyperref}
\hypersetup{
    colorlinks,
    citecolor=black,
    filecolor=blue,
    linkcolor=red,
    urlcolor=blue
}
\usepackage{listings}
\usepackage{color}

\definecolor{mygreen}{rgb}{0,0.6,0}
\definecolor{mygray}{rgb}{0.5,0.5,0.5}
\definecolor{mymauve}{rgb}{0.58,0,0.82}
\lstset{ %
  backgroundcolor=\color{white},   % choose the background color; you must add 
  %\usepackage{color} or \usepackage{xcolor}; should come as last argument
  basicstyle=\footnotesize,        % the size of the fonts that are used for 
  %the code
  breakatwhitespace=false,         % sets if automatic breaks should only 
  %happen at whitespace
  breaklines=true,                 % sets automatic line breaking
  captionpos=b,                    % sets the caption-position to bottom
  commentstyle=\color{mygreen},    % comment style
  deletekeywords={...},            % if you want to delete keywords from the 
  %given language
  escapeinside={\%*}{*)},          % if you want to add LaTeX within your code
  extendedchars=true,              % lets you use non-ASCII characters; for 
  %8-bits encodings only, does not work with UTF-8
  frame=single,	                   % adds a frame around the code
  keepspaces=true,                 % keeps spaces in text, useful for keeping 
  %indentation of code (possibly needs columns=flexible)
  keywordstyle=\color{blue},       % keyword style
  language=ML,                 % the language of the code
  morekeywords={*,...},            % if you want to add more keywords to the set
  numbers=left,                    % where to put the line-numbers; possible 
  %values are (none, left, right)
  numbersep=5pt,                   % how far the line-numbers are from the code
  numberstyle=\tiny\color{mygray}, % the style that is used for the line-numbers
  rulecolor=\color{black},         % if not set, the frame-color may be changed 
  %on line-breaks within not-black text (e.g. comments (green here))
  showspaces=false,                % show spaces everywhere adding particular 
  %underscores; it overrides 'showstringspaces'
  showstringspaces=false,          % underline spaces within strings only
  showtabs=false,                  % show tabs within strings adding particular 
  %underscores
  stepnumber=2,                    % the step between two line-numbers. If it's 
  %1, each line will be numbered
  stringstyle=\color{mymauve},     % string literal style
  tabsize=2,	                   % sets default tabsize to 2 spaces
  title=\lstname                   % show the filename of files included with 
  %\lstinputlisting; also try caption instead of title
}
%% Comments

\usepackage{color}

\newif\ifcomments\commentstrue

\ifcomments
\newcommand{\authornote}[3]{\textcolor{#1}{[#3 ---#2]}}
\newcommand{\todo}[1]{\textcolor{red}{[TODO: #1]}}
\else
\newcommand{\authornote}[3]{}
\newcommand{\todo}[1]{}
\fi

\newcommand{\wss}[1]{\authornote{blue}{SS}{#1}}
\newcommand{\als}[1]{\authornote{magenta}{AS}{#1}}


\usepackage{xspace}

\newcommand{\progname}{the RKF45 Generator\xspace}

\newcounter{acnum}
\newcommand{\actheacnum}{AC\theacnum}
\newcommand{\acref}[1]{AC\ref{#1}}

\newcounter{ucnum}
\newcommand{\uctheucnum}{UC\theucnum}
\newcommand{\uref}[1]{UC\ref{#1}}

\newcounter{mnum}
\newcommand{\mthemnum}{M\themnum}
\newcommand{\mref}[1]{M\ref{#1}}

\begin{document}

\title{Design: Runge-Kutta (RK) Generator} 
%\author{Alexander Schaap}
\date{\today}

\maketitle

\pagenumbering{roman}

\section{Revision History}

The latest version can be found at \url{https://github.com/aschaap/cas741}.\\

\noindent
\begin{tabularx}{\textwidth}{p{3cm}p{2cm}X}
\toprule {\bf Date} & {\bf Version} & {\bf Notes}\\
\midrule
November 3 & 1.0 & Initial Draft\\
November 8 & 1.1 & Processed feedback, continued writing\\
November 29 & 1.2 & Processed feedback, added interface specifications\\
December 7 & 1.3 & Processed Paul's feedback\\
December 18 & 1.4 & Final version, processed more feedback\\
\bottomrule
\end{tabularx}

\newpage

\section{Symbols, Abbreviations \& Acronyms}
See the \href{../SRS/CA.pdf#ssec:symbols}{Table of Symbols} in the SRS at 
\url{https://github.com/aschaap/cas741}.


\newpage

\tableofcontents

%\listoftables
%
%\listoffigures

\newpage

\pagenumbering{arabic}

\section{Introduction}

\wss{Reusing this introduction does not make any sense for your project.  Please
  replace with an introduction suitable for your design document.}
\als{I have adapted it to illustrate the need for describing binding times.}

Decomposing a system into modules is a commonly accepted approach to developing
software.  A module is a work assignment for a programmer or programming
team~\citep{ParnasEtAl1984}.  We advocate a decomposition
based on the principle of information hiding~\citep{Parnas1972a}.  This
principle supports design for change, because the ``secrets'' that each module
hides represent likely future changes.  Design for change is valuable in SC,
where modifications are frequent, especially during initial development as the
solution space is explored. However, for this particular project, a 
decomposition into modules is not really useful because it is essentially a 
single module to begin with.

For multi-stage programming (MSP), there is another dimension in which 
decomposition can occur: binding times. While traditional software gives the 
option of binding variables to values at either compile time or runtime, MSP 
introduces the concept of staging, or delaying execution of code annotated as 
such. This gives the programmer more possible binding times. Delaying execution 
allows one to combine fragments of code into larger ones, which will be called 
generation time for clarity. This occurs after compilation time. Runtime is the 
moment the generated code is executed. Generators-in-generators are not 
considered because they are overkill for the problem of solving ODEs. 

An analysis of anticipated and unlikely changes is still relevant; however, 
this will not translate to modules. It will affect the individual module 
decomposition (into functions), which merits consideration on its 
own, as illustrated by \cite{VanHilst:1996:DCD:250707.239109}. Decomposing by 
binding time is the way this single-module program is decomposed. Therefore, 
the following anticipated changes have been identified:
\begin{description}
  \item[\refstepcounter{acnum} \actheacnum \label{acAlgorithm}:] The algorithm 
  used to solve the given IVP.
  \item[\refstepcounter{acnum} \actheacnum \label{acImplementation}:] The 
  implementation of the algorithms used to solve the IVP.
  \item [\refstepcounter{acnum} \actheacnum \label{acInterpolation}:] Adding 
  interpolation of the solution provided by the specified algorithm.
\end{description}

As well as the following unlikely changes:
\begin{description}
  \item[\refstepcounter{ucnum} \uctheucnum \label{ucOutput}:] Output format 
  (generated code).
  \item[\refstepcounter{ucnum} \uctheucnum \label{ucInput}:] The way the driver 
  program invokes the software.
\end{description}

The potential readers of this document are as follows:

\begin{itemize}
\item New project members: This document can be a guide for a new project member
  to easily understand the overall structure and quickly find the
  relevant modules they are searching for.
\item Maintainers: The hierarchical structure of the module guide improves the
  maintainers' understanding when they need to make changes to the system. It is
  important for a maintainer to update the relevant sections of the document
  after changes have been made.
\item Designers: Once the module guide has been written, it can be used to
  check for consistency, feasibility and flexibility. Designers can verify the
  system in various ways, such as consistency among modules, feasibility of the
  decomposition, and flexibility of the design.
\end{itemize}

The rest of the document is organized as follows. Section~\ref{sec:Notation} 
describes the notation used throughout this document. 
Section~\ref{sec:Binding_Times} illustrates the alternative criterion used to 
decompose this program.
%Section~\ref{SecChange} lists the anticipated and unlikely changes of the 
%software requirements. 
%Section \ref{SecMH} \wss{missing this section} summarizes the module 
%decomposition that was constructed according to the likely changes. 
Section~\ref{SecConnection} specifies the connections between the software 
requirements and the modules. Section~\ref{SecMD} gives a detailed description 
of the modules. Section~\ref{SecTM} includes two traceability matrices. 
One checks the completeness of the design against the requirements provided in 
the SRS. The other shows the relation between anticipated changes and the 
modules. Section~\ref{SecUse} describes the use relation between modules.

\section{Notation}\label{sec:Notation}


The structure of the MIS for modules comes from \citet{HoffmanAndStrooper1995},
with the addition that template modules have been adapted from
\cite{GhezziEtAl2003}.  The mathematical notation comes from Chapter 3 of
\citet{HoffmanAndStrooper1995}.  For instance, the symbol := is used for a
multiple assignment statement and conditional rules follow the form $(c_1
\Rightarrow r_1 | c_2 \Rightarrow r_2 | ... | c_n \Rightarrow r_n )$.

The following table summarizes the primitive data types used by \progname. 

\begin{center}
  \renewcommand{\arraystretch}{1.2}
  \noindent 
  \begin{tabular}{l l p{7.5cm}} 
    \toprule 
    \textbf{Data Type} & \textbf{Notation} & \textbf{Description}\\ 
    \midrule
    character & char & a single symbol or digit\\
    natural number & $\mathbb{N}$ & a number without a fractional component in 
    [1, $\infty$) \\
    natural number & $\mathbb{N}^0$ & a number without a fractional component 
    in 
    [0, $\infty$) \\
    real & $\mathbb{R}$ & any number in (-$\infty$, $\infty$)\\
    \bottomrule
  \end{tabular} 
\end{center}

\wss{I see complex numbers later.  You should put this type in your table too.}

\noindent
The specification of \progname \ uses some derived data types: strings, 
sequences and functions. 
Sequences are lists filled with elements of the same data type.
Strings are sequences of characters. 
% Tuples contain a list of values, potentially of different types.
In addition, \progname uses functions, which are defined by the data types of 
their inputs and outputs. Functions can also be arguments to other functions 
(e.g. the ODE is provided to the generator). 
Local functions are described by giving their type signature followed by their 
specification.


%\section{Anticipated and Unlikely Changes} \label{SecChange}
%
%This section lists possible changes to the system. According to the likeliness
%of the change, the possible changes are classified into two
%categories. Anticipated changes are listed in Section \ref{SecAchange}, and
%unlikely changes are listed in Section \ref{SecUchange}.
%
\wss{These ideas still apply, but rather than separate sections, maybe you 
could just summarize this information in your revised introduction.}
\als{I've shortened and moved anticipated and unlikely changes to the intro.}
%
%\subsection{Anticipated Changes} \label{SecAchange}
%
%Anticipated changes are the source of the information that is to be hidden
%inside the modules. Ideally, changing one of the anticipated changes will only
%require changing the one module that hides the associated decision. The 
%approach
%adapted here is called design for
%change.
%
%\begin{description}
%\item[\refstepcounter{acnum} \actheacnum \label{acAlgorithm}:] The algorithm 
%used to solve the given IVP.
%\item[\refstepcounter{acnum} \actheacnum \label{acImplementation}:] The 
%implementation of the algorithms used to solve the IVP.
%\item ...
%\end{description}
%
%\subsection{Unlikely Changes} \label{SecUchange}
%
%The module design should be as general as possible. However, a general system 
%is
%more complex. Sometimes this complexity is not necessary. Fixing some design
%decisions at the system architecture stage can simplify the software design. If
%these decision should later need to be changed, then many parts of the design
%will potentially need to be modified. Hence, it is not intended that these
%decisions will be changed.
%
%\begin{description}
%\item[\refstepcounter{ucnum} \uctheucnum \label{ucOutput}:] Output format 
%(generated code).
%\item[\refstepcounter{ucnum} \uctheucnum \label{ucInput}:] The way the driver 
%invokes the software.
%\item ...
%\end{description}

%\section{Module Hierarchy} \label{SecMH}
%
%This section provides an overview of the module design. Modules are summarized
%in a hierarchy decomposed by secrets in Table \ref{TblMH}. The modules listed
%below, which are leaves in the hierarchy tree, are the modules that will
%actually be implemented.
%
%\begin{description}
%\item [\refstepcounter{mnum} \mthemnum \label{mHH}:] Hardware-Hiding Module
%\item ...
%\end{description}
%
%
%\begin{table}[h!]
%\centering
%\begin{tabular}{p{0.3\textwidth} p{0.6\textwidth}}
%\toprule
%\textbf{Level 1} & \textbf{Level 2}\\
%\midrule
%
%{Hardware-Hiding Module} & ~ \\
%\midrule
%
%\multirow{7}{0.3\textwidth}{Behaviour-Hiding Module} & ?\\
%& ?\\
%& ?\\
%& ?\\
%& ?\\
%& ?\\
%& ?\\ 
%& ?\\
%\midrule
%
%\multirow{3}{0.3\textwidth}{Software Decision Module} & {?}\\
%& ?\\
%& ?\\
%\bottomrule
%
%\end{tabular}
%\caption{Module Hierarchy}
%\label{TblMH}
%\end{table}

\section{Binding Times}\label{sec:Binding_Times}

This section will demonstrate the difference between binding times via the 
oft-used power example. Rather than dividing the program into modules, a 
decomposition by binding times is given. Such a decomposition is more 
appropriate when multi-stage programming (MSP) is employed, something 
\cite{Parnas1972a} was unaware of at that time, due to Multi-stage programming 
being much newer. 
%In terms of work assignments, perhaps the functions that make up the program 
%could be assigned to different programmers.
\subsection{Example}
Consider the following example\footnote{Adapted from 
\url{http://okmij.org/ftp/ML/MetaOCaml.html}} for calculating any $x^n$:
\begin{lstlisting}
let square x = x * x
let rec power n x =
  if n = 0 then 1
  else if n mod 2 = 0 then square (power (n/2) x)
  else x * (power (n-1) x)
(* val power : int -> int -> int = <fun> *)
\end{lstlisting}

(The comment below the code shows the reply of the MetaOCaml interactive 
prompt, called the ``toplevel'', which contains the types of the power 
function.)

Assuming $x^7$ is a common operation, it would be useful to define
\begin{lstlisting}
let power7 x = power 7 x
\end{lstlisting}
to give the \emph{partial application} a name and use it throughout the code. 
Partial application means 
that when giving a function some but not all the arguments it expects, one 
creates a new function that only expects the remaining arguments. In the case 
of \lstinline|power7|, the resulting function type will be 
\lstinline|int -> int|.

In MetaOCaml, one can also create a specialized instance of the power function 
for a particular value of $n$. However, in the case of MetaOCaml, one can 
generate code which accepts any $x$ and computes $x^n$. To do this, the
\lstinline|power| function must be annotated such that computations which can 
be performed at generation time are distinguished from those that must be 
performed at run time. Because information such as $n$ is available at 
right away and certain not to change in the future, any computation that 
only depends on $n$ may be performed when the code is being generated. In 
contrast, information such as $x$ is not known at generation time and subject 
to change, so any computation that needs it will have to be postponed until the 
generated code is actually run (when x is given).
\begin{lstlisting}
let rec spower n x =
  if n = 0 then .<1>.
  else if n mod 2 = 0 then .<square .~(spower (n/2) x)>.
  else .<.~x * .~(spower (n-1) x)>.;;
(* val spower : int -> int code -> int code = <fun> *)
\end{lstlisting}
Two annotations, or staging constructs, specific to MetaOCaml appear in the 
code above. These are brackets \lstinline|.< e >.| and escape \lstinline|.~e|. 
Brackets \lstinline|.< e >.| `quasi-quote' the expression e, annotating it as 
computed later. Escape \lstinline|.~e|, which must be used within brackets, 
tells that e is computed now but produces the result for later. That result, 
the code, is inserted (spliced-in) into the containing bracket. The inferred 
type of \lstinline|spower| is different. The 
result is no longer an \lstinline|int|, but \lstinline|int code| -- the code of 
expressions that compute an \lstinline|int|. The type of \lstinline|spower| 
spells out which argument is received now, and which later: the future-stage 
argument has the \lstinline|code| type. To specialize \lstinline|spower| to 7, 
one can define
\begin{lstlisting}
let spower7_code = .<fun x -> .~(spower 7 .<x>.)>.;;
(*
val spower7_code : (int -> int) code = .<
  fun x_1  -> x_1 * ((* CSP square *) (x_1 * ((* CSP square *) (x_1 * 
    1))))>.
*)
\end{lstlisting}
and obtain the code of a function that expects any $x$ and returns $x^7$. Code, 
even of functions, can be printed, which is what MetaOCaml toplevel 
(interactive prompt) just did. The print-out contains so-called cross-stage 
persistent values, or CSP, which `quote' present-stage values such as square to 
be used later. One may think of CSP as references to `external libraries' -- 
only in our case the program acts as a library for the code it generates. If 
square were defined in a separate file such as \texttt{sq.ml}, then its 
occurrence in \lstinline|spower7_code| would have been printed as 
\lstinline|Sq.square|.

To use the specialized $x^7$ now, in our code, we should `run' 
\lstinline|spower7_code|, applying the function 
\lstinline|Runcode.run : 'a code -> 'a|. The function compiles the code and 
links it back to our program. Runcode.run is aliased to the prefix operation 
(!.):
\begin{lstlisting}
open Runcode
let spower7 = !. spower7_code
(* val spower7 : int -> int = <fun> *)
\end{lstlisting}

The specialized \lstinline|spower7| has the same type as the partially applied 
\lstinline|power7| above. They behave identically. However, 
\lstinline|power7 x| will do the recursion on $n$, checking $n$'s parity, etc. 
In contrast, the specialized \lstinline|spower7| has no recursion (as can be 
seen from \lstinline|spower7_code|). All operations on n have been done when 
the \lstinline|spower7_code| was computed, producing the straight-line code 
\lstinline|spower7| that only multiplies $x$.

(MetaOCaml supports an arbitrary number of later stages, letting one write not 
only code generators but also generators of code generators, etc.)

\wss{I like the above summary.  I would like you to use the same systematic
  steps for presenting your code.}
\als{I'm afraid this is challenging.}

\subsection{Compile-Time}\label{ssec:compile-time}
As the name implies, this is when the code is compiled. Code generation could 
occur at this time, but this will not happen for the RK generator. Instead, 
generation will occur when the \lstinline[language=ML]|odesolve| entry-point 
function is called. 
This is the only function exposed to the driver program initially. (Generated 
code will be available after calling \lstinline[language=ML]|odesolve|)

The generator and its RK methods will be completely defined at this time. There 
are no inputs to it at this time, because it is effectively a library. However, 
the interface should satisfy the following:

\subsubsection{Syntax}
Syntax of the access programs available after this binding time.

\paragraph{Exported Access Programs}

\begin{center}
  \begin{tabular}{p{1.6cm} p{5.8cm} p{3.2cm} p{4.2cm}}
    \hline
    \textbf{Name} & \textbf{In} & \textbf{Out} & \textbf{Exceptions} \\
    \hline
    odesolve &
    $ [t_0,t_N] : \mathbb{R} \times \mathbb{R} , k : \mathbb{N} , 
    {\bf x_0} : \mathbb{C}^n , {\bf f} : \mathbb{R} \times \mathbb{C}^n 
    \rightarrow \mathbb{C}^n $ &
    $(\mathbb{N}^0 \times \mathbb{R} \rightarrow \mathbb{R})$ \texttt{code} & 
    LESS\_THAN\_2\_KNOTS \\
    \hline
  \end{tabular}
\end{center}

\wss{You have complex numbers for your ode?  This is fine, but please make sure
  that your SRS documentation reflects this decision.  (I haven't checked to see
  what the SRS says.}
\wss{What does the type of the output code imply?  I'm not sure what the natural
  number is for?  I would have thought the output would be $\mathbb{R}
  \rightarrow \mathbb{C}^n$ (a function from time to the complex numbers
  defining the current values of the state variables).}
For the sake of clarity and easier reference from the semantics portion, the 
identifiers match most of the data definitions from the SRS. Note that $k$ is 
the exception; this is the number of knots (to divide the interval into to 
determine the stepsize $h$ from the SRS). \wss{Have you updated the SRS to
  mention knots?  The use of splines to obtain values between the ode solver
  points is something that wasn't previously clear in your documentation.}

\subsubsection{Semantics}

Semantics of the access program(s) available after this binding time.

\paragraph{State Variables}

none

\paragraph{Access Routine Semantics}

\noindent odesolve():
\begin{itemize}
  \item transition: N/A
  \item output: $ out := (\mathbb{N}^0 \times \mathbb{R} \rightarrow 
  \mathbb{R})$ 
  \texttt{code}
  \begin{itemize}
    \item Code that contains a function that will provide a solution for points 
    such that point $p = t_0 + m \times h$ based on precomputed knowledge
    embedded in it by the generator. \wss{This doesn't actually make things any
      clearer.  I guess the natural number is $m$?  What happened to the complex
      numbers?  Your initial values were a sequence of values, corresponding to
      state variables $x_1$, $x_2$, ... $x_n$.  What happened to this sequence?}
  \end{itemize}
  \item exception: $exc := (k < 2 \Rightarrow$ LESS\_THAN\_2\_KNOTS)
  \begin{itemize}
    \item The number of knots should be at least two, which would correspond to 
    one at each end of the interval, and the step size $h$ being equal to the 
    size of the interval. Behaviour for less than two knots is undefined.
  \end{itemize}
\end{itemize}

\subsubsection{Local Functions} \wss{I don't see any complex numbers here.  Was
  there earlier introduction a mistake, or am I missing something?}

These are other functions utilized by \lstinline[language=ML]|odesolve| to 
generate its output. They also need to be implemented by the programmer(s).\\
\\
evalrk45: $\mathbb{R} \times \mathbb{R} \times \mathbb{R}^n \times (\mathbb{R} 
\times \mathbb{R}^n \rightarrow \mathbb{R}^n)$ $\rightarrow$ $\mathbb{R}^n$\\
evalrk45($t_k$,$h$, ${\bf x}_k$, ${\bf f}$) $\equiv$ IM1

This function corresponds to Instance Model 1 in the SRS.\\
\\
evalrk23: $\mathbb{R} \times \mathbb{R} \times \mathbb{R}^n \times (\mathbb{R} 
\times \mathbb{R}^n \rightarrow \mathbb{R}^n)$ $\rightarrow$ $\mathbb{R}^n$\\
evalrk23($t_k$,$h$, ${\bf x}_k$, ${\bf f}$) $\equiv$ IM2

This function corresponds to Instance Model 2 in the SRS.\\
\\
rk: $\mathbb{R}^n \times \mathbb{R}^n \times 
(\mathbb{R} \times \mathbb{R}^n \rightarrow \mathbb{R}^n) \times
(\mathbb{R} \times \mathbb{R} \times \mathbb{R}^n \times (\mathbb{R} 
\times \mathbb{R}^n \rightarrow \mathbb{R}^n) \rightarrow \mathbb{R}^n) 
\rightarrow (\mathbb{R}^n)^k$\\
rk($karray$, ${\bf x}_k$, ${\bf f}$, $evalrk$) $\equiv$ $result : (\forall k | 
k 
\in karray : result_i = evalrk (k_i,h, {\bf x}_k, {\bf f}))$

This function calls \lstinline[language=ML]|evalrk45| or 
\lstinline[language=ML]|evalrk23| on every point in the interval that happens 
to be a multiple of the step size ($h$). $karray$ is a sequence of size $k$ 
containing all the points for 
which to run the chosen RK method. $result$ is a two-dimensional 
sequence of size $n \times k$ containing the result of running the RK method on 
each point. $evalrk$ is either the \lstinline[language=ML]|evalrk45| or 
\lstinline[language=ML]|evalrk23| function.\\
\\
construct: $\mathbb{R} \times \mathbb{R} \times \mathbb{R}^n \times 
(\mathbb{R}^n)^k \times \mathbb{N} \rightarrow (\mathbb{N} \times \mathbb{R} 
\rightarrow (string \lor \mathbb{R}))$ code\\
construct($t_0$,$t_N$,$karray$,$result$) $\equiv$ anonymous function(i,t) 
$\equiv$ $(result_t)_i$ $\lor$ OUT\_OF\_RANGE

This function takes the result of \lstinline[language=ML]|rk| and produces a 
function that will return specific result values for specific inputs \wss{spell
  check} within the 
interval. $karray$ and $result$ are the same as above.

\subsection{Generation Time}\label{ssec:generation-time}
This is part of the run-time of the driver program, but from the point of view 
of this library, separating the moment the code for a particular IVP is 
generated and the moment(s) it is executed is useful. Since there is only one 
access function available to the driver program, this is what happens when this 
function is called.

The ODE, interval, step size, desired RK method and initial value are known at 
generation time. 
Since these will not change when looking for solutions given a particular point 
on the interval for which the algorithm is solving, the IVP can be solved at 
this time using the entry-point function created at compile-time.

Note that the generated code consists of a function that is not explicitly 
named; this is up to the driver program. Therefore, the function name is 
represented by ``anonymous function''.
%The type of the ODE is \lstinline|float -> float array -> float array|. The 
%types for the range, step size and initial value are 
%\lstinline|(float * float)|, \lstinline|float|, and \lstinline|float array| 
%respectively.
%The type of the function that is returned should be (close to) 
%\lstinline|(float -> float) code|.
\subsection{Syntax}

\subsubsection{Exported Access Programs}

\begin{center}
  \begin{tabular}{p{5cm} p{3cm} p{3cm} p{3cm}}
    \hline
    \textbf{Name} & \textbf{In} & \textbf{Out} & \textbf{Exceptions} \\
    \hline
    N/A - anonymous function & i : $\mathbb{N}_0, t : \mathbb{R}$ & $ x_i(t) : 
    \mathbb{R}$ & NOT\_IN\_RANGE \\
    \hline
  \end{tabular}
\end{center}

\subsection{Semantics}
Semantics of the access program(s) available after this binding time.

\subsubsection{State Variables}
None.

\subsubsection{Access Routine Semantics}

\noindent ``anonymous function''():
\begin{itemize}
  \item transition: none
  \item output: $out := x_{i}(t) : \mathbb{R}$
  \begin{itemize}
    \item A numerical value for the particular point and ODE given as input
  \end{itemize}
  \item exception: $exc := (t < t_0 \lor t_N < t \Rightarrow$ NOT\_IN\_RANGE)
\end{itemize}
\subsection{Run-Time}\label{ssec:run-time}
This is also part of the driver program's run-time, but from the point of view 
of the RK generator, this is (one of) the moment(s) when the generated code is 
run.

The function that will return values for any input within the specified 
interval (exported at generation time) will execute at run-time.

%As shown in the previous subsection, the input will be a \lstinline|float|, 
%and 
%the return type is another \lstinline|float|.



\section{Connection Between Requirements and Design} \label{SecConnection}

N/A. While the commonality analysis has clear implications, the requirements 
are not explicitly recorded anywhere. 

\section{Module Decomposition} \label{SecMD}
N/A. The program is broken down by binding times, rather than modules, due to 
the use of multi-stage programming via MetaOCaml.
%
%Modules are decomposed according to the principle of ``information hiding''
%proposed by \citet{ParnasEtAl1984}. The \emph{Secrets} field in a module
%decomposition is a brief statement of the design decision hidden by the
%module. The \emph{Services} field specifies \emph{what} the module will do
%without documenting \emph{how} to do it. For each module, a suggestion for the
%implementing software is given under the \emph{Implemented By} title. If the
%entry is \emph{OS}, this means that the module is provided by the operating
%system or by standard programming language libraries.  Also indicate if the
%module will be implemented specifically for the software.
%
%Only the leaf modules in the
%hierarchy have to be implemented. If a dash (\emph{--}) is shown, this means
%that the module is not a leaf and will not have to be implemented. Whether or
%not this module is implemented depends on the programming language
%selected.
%
%\subsection{Hardware Hiding Modules (\mref{mHH})}
%
%\begin{description}
%\item[Secrets:]The data structure and algorithm used to implement the virtual
%  hardware.
%\item[Services:]Serves as a virtual hardware used by the rest of the
%  system. This module provides the interface between the hardware and the
%  software. So, the system can use it to display outputs or to accept inputs.
%\item[Implemented By:] OS
%\end{description}
%
%\subsection{Behaviour-Hiding Module}
%
%\begin{description}
%\item[Secrets:]The contents of the required behaviours.
%\item[Services:]Includes programs that provide externally visible behaviour of
%  the system as specified in the software requirements specification (SRS)
%  documents. This module serves as a communication layer between the
%  hardware-hiding module and the software decision module. The programs in this
%  module will need to change if there are changes in the SRS.
%\item[Implemented By:] --
%\end{description}
%
%\subsubsection{Input Format Module (\mref{mInput})}
%
%\begin{description}
%\item[Secrets:]The format and structure of the input data.
%\item[Services:]Converts the input data into the data structure used by the
%  input parameters module.
%\item[Implemented By:] [Your Program Name Here]
%\end{description}
%
%\subsubsection{Etc.}
%
%
%\subsection{Software Decision Module}
%
%\begin{description}
%\item[Secrets:] The design decision based on mathematical theorems, physical
%  facts, or programming considerations. The secrets of this module are
%  \emph{not} described in the SRS.
%\item[Services:] Includes data structure and algorithms used in the system that
%  do not provide direct interaction with the user. 
%  % Changes in these modules are more likely to be motivated by a desire to
%  % improve performance than by externally imposed changes.
%\item[Implemented By:] --
%\end{description}
%
%\subsubsection{Etc.}
%
\section{Traceability Matrix} \label{SecTM}
N/A.
%
%This section shows two traceability matrices: between the modules and the
%requirements and between the modules and the anticipated changes.
%
%% the table should use mref, the requirements should be named, use something
%% like fref
%\begin{table}[H]
%\centering
%\begin{tabular}{p{0.2\textwidth} p{0.6\textwidth}}
%\toprule
%\textbf{Req.} & \textbf{Modules}\\
%\midrule
%R1 & \mref{mHH}, \mref{mInput}, \mref{mParams}, \mref{mControl}\\
%R2 & \mref{mInput}, \mref{mParams}\\
%R3 & \mref{mVerify}\\
%R4 & \mref{mOutput}, \mref{mControl}\\
%R5 & \mref{mOutput}, \mref{mODEs}, \mref{mControl}, \mref{mSeqDS}, 
%\mref{mSolver}, \mref{mPlot}\\
%R6 & \mref{mOutput}, \mref{mODEs}, \mref{mControl}, \mref{mSeqDS}, 
%\mref{mSolver}, \mref{mPlot}\\
%R7 & \mref{mOutput}, \mref{mEnergy}, \mref{mControl}, \mref{mSeqDS}, 
%\mref{mPlot}\\
%R8 & \mref{mOutput}, \mref{mEnergy}, \mref{mControl}, \mref{mSeqDS}, 
%\mref{mPlot}\\
%R9 & \mref{mVerifyOut}\\
%R10 & \mref{mOutput}, \mref{mODEs}, \mref{mControl}\\
%R11 & \mref{mOutput}, \mref{mODEs}, \mref{mEnergy}, \mref{mControl}\\
%\bottomrule
%\end{tabular}
%\caption{Trace Between Requirements and Modules}
%\label{TblRT}
%\end{table}
%
%\begin{table}[H]
%\centering
%\begin{tabular}{p{0.2\textwidth} p{0.6\textwidth}}
%\toprule
%\textbf{AC} & \textbf{Modules}\\
%\midrule
%\acref{acHardware} & \mref{mHH}\\
%\acref{acInput} & \mref{mInput}\\
%\acref{acParams} & \mref{mParams}\\
%\acref{acVerify} & \mref{mVerify}\\
%\acref{acOutput} & \mref{mOutput}\\
%\acref{acVerifyOut} & \mref{mVerifyOut}\\
%\acref{acODEs} & \mref{mODEs}\\
%\acref{acEnergy} & \mref{mEnergy}\\
%\acref{acControl} & \mref{mControl}\\
%\acref{acSeqDS} & \mref{mSeqDS}\\
%\acref{acSolver} & \mref{mSolver}\\
%\acref{acPlot} & \mref{mPlot}\\
%\bottomrule
%\end{tabular}
%\caption{Trace Between Anticipated Changes and Modules}
%\label{TblACT}
%\end{table}
%
\section{Use Hierarchy Between Modules} \label{SecUse}
N/A.
%
%In this section, the uses hierarchy between modules is
%provided. \citet{Parnas1978} said of two programs A and B that A {\em uses} B 
%if
%correct execution of B may be necessary for A to complete the task described in
%its specification. That is, A {\em uses} B if there exist situations in which
%the correct functioning of A depends upon the availability of a correct
%implementation of B.  Figure \ref{FigUH} illustrates the use relation between
%the modules. It can be seen that the graph is a directed acyclic graph
%(DAG). Each level of the hierarchy offers a testable and usable subset of the
%system, and modules in the higher level of the hierarchy are essentially 
%simpler
%because they use modules from the lower levels.
%
%\begin{figure}[H]
%\centering
%%\includegraphics[width=0.7\textwidth]{UsesHierarchy.png}
%\caption{Use hierarchy among modules}
%\label{FigUH}
%\end{figure}

%\section*{References}

\wss{I had to delete the Parnas1972a to get bibtex to work.}

\bibliographystyle {plainnat}
\bibliography {Design,../References}

\wss{What I was hoping to see was a transition from the unstaged code to the
  staged version, like you did for the square function, and like what was done
  for the vector norm in the ggk paper.  At the very least showing the unstaged
  code shows the RK algorithm.}  \wss{Do you think someone, who knows MetaOcaml,
  could implement the ode solver generator from the design specification.  I
  feel like information is missing.  How do the local functions fit with the
  odesolve function?  The reader has to piece that together for themselves?}
\wss{Maybe you could include more code snippets in your design.  As we briefly
  discussed a literate document would be a helpful approach here.}
\end{document}