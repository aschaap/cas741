\documentclass[12pt, titlepage]{article}

\usepackage{fullpage}
\usepackage[round]{natbib}
\usepackage{multirow}
\usepackage{booktabs}
\usepackage{tabularx}
\usepackage{graphicx}
\usepackage{float}
\usepackage{hyperref}
\hypersetup{
    colorlinks,
    citecolor=black,
    filecolor=black,
    linkcolor=red,
    urlcolor=blue
}
\usepackage{listings}
\usepackage{color}

\definecolor{mygreen}{rgb}{0,0.6,0}
\definecolor{mygray}{rgb}{0.5,0.5,0.5}
\definecolor{mymauve}{rgb}{0.58,0,0.82}
\lstset{ %
  backgroundcolor=\color{white},   % choose the background color; you must add 
  %\usepackage{color} or \usepackage{xcolor}; should come as last argument
  basicstyle=\footnotesize,        % the size of the fonts that are used for 
  %the code
  breakatwhitespace=false,         % sets if automatic breaks should only 
  %happen at whitespace
  breaklines=true,                 % sets automatic line breaking
  captionpos=b,                    % sets the caption-position to bottom
  commentstyle=\color{mygreen},    % comment style
  deletekeywords={...},            % if you want to delete keywords from the 
  %given language
  escapeinside={\%*}{*)},          % if you want to add LaTeX within your code
  extendedchars=true,              % lets you use non-ASCII characters; for 
  %8-bits encodings only, does not work with UTF-8
  frame=single,	                   % adds a frame around the code
  keepspaces=true,                 % keeps spaces in text, useful for keeping 
  %indentation of code (possibly needs columns=flexible)
  keywordstyle=\color{blue},       % keyword style
  language=ML,                 % the language of the code
  morekeywords={*,...},            % if you want to add more keywords to the set
  numbers=left,                    % where to put the line-numbers; possible 
  %values are (none, left, right)
  numbersep=5pt,                   % how far the line-numbers are from the code
  numberstyle=\tiny\color{mygray}, % the style that is used for the line-numbers
  rulecolor=\color{black},         % if not set, the frame-color may be changed 
  %on line-breaks within not-black text (e.g. comments (green here))
  showspaces=false,                % show spaces everywhere adding particular 
  %underscores; it overrides 'showstringspaces'
  showstringspaces=false,          % underline spaces within strings only
  showtabs=false,                  % show tabs within strings adding particular 
  %underscores
  stepnumber=2,                    % the step between two line-numbers. If it's 
  %1, each line will be numbered
  stringstyle=\color{mymauve},     % string literal style
  tabsize=2,	                   % sets default tabsize to 2 spaces
  title=\lstname                   % show the filename of files included with 
  %\lstinputlisting; also try caption instead of title
}
%% Comments

\usepackage{color}

\newif\ifcomments\commentstrue

\ifcomments
\newcommand{\authornote}[3]{\textcolor{#1}{[#3 ---#2]}}
\newcommand{\todo}[1]{\textcolor{red}{[TODO: #1]}}
\else
\newcommand{\authornote}[3]{}
\newcommand{\todo}[1]{}
\fi

\newcommand{\wss}[1]{\authornote{blue}{SS}{#1}}
\newcommand{\als}[1]{\authornote{magenta}{AS}{#1}}


\newcounter{acnum}
\newcommand{\actheacnum}{AC\theacnum}
\newcommand{\acref}[1]{AC\ref{#1}}

\newcounter{ucnum}
\newcommand{\uctheucnum}{UC\theucnum}
\newcommand{\uref}[1]{UC\ref{#1}}

\newcounter{mnum}
\newcommand{\mthemnum}{M\themnum}
\newcommand{\mref}[1]{M\ref{#1}}

\begin{document}

\title{Module Guide: Project Title} 
\author{Author Name}
\date{\today}

\maketitle

\pagenumbering{roman}

\section{Revision History}

\begin{tabularx}{\textwidth}{p{3cm}p{2cm}X}
\toprule {\bf Date} & {\bf Version} & {\bf Notes}\\
\midrule
November 3 & 1.0 & Initial Draft\\
November 8 & 1.1 & Processed feedback, continued writing\\
\bottomrule
\end{tabularx}

\newpage

\tableofcontents

%\listoftables
%
%\listoffigures

\newpage

\pagenumbering{arabic}

\section{Introduction}

\wss{Reusing this introduction does not make any sense for your project.  Please
  replace with an introduction suitable for your design document.}

Decomposing a system into modules is a commonly accepted approach to developing
software.  A module is a work assignment for a programmer or programming
team~\citep{ParnasEtAl1984}.  We advocate a decomposition
based on the principle of information hiding~\citep{Parnas1972a}.  This
principle supports design for change, because the ``secrets'' that each module
hides represent likely future changes.  Design for change is valuable in SC,
where modifications are frequent, especially during initial development as the
solution space is explored.  

Our design follows the rules laid out by \citet{ParnasEtAl1984}, as follows:
\begin{itemize}
\item System details that are likely to change independently should be the
  secrets of separate modules.
\item Each data structure is used in only one module.
\item Any other program that requires information stored in a module's data
  structures must obtain it by calling access programs belonging to that module.
\end{itemize}

After completing the first stage of the design, the Software Requirements
Specification (SRS), the Module Guide (MG) is developed~\citep{ParnasEtAl1984}. The MG
specifies the modular structure of the system and is intended to allow both
designers and maintainers to easily identify the parts of the software.  The
potential readers of this document are as follows:

\begin{itemize}
\item New project members: This document can be a guide for a new project member
  to easily understand the overall structure and quickly find the
  relevant modules they are searching for.
\item Maintainers: The hierarchical structure of the module guide improves the
  maintainers' understanding when they need to make changes to the system. It is
  important for a maintainer to update the relevant sections of the document
  after changes have been made.
\item Designers: Once the module guide has been written, it can be used to
  check for consistency, feasibility and flexibility. Designers can verify the
  system in various ways, such as consistency among modules, feasibility of the
  decomposition, and flexibility of the design.
\end{itemize}

The rest of the document is organized as follows. Section \ref{SecChange} lists
the anticipated and unlikely changes of the software requirements. Section
\ref{SecMH} \wss{missing this section} summarizes the module decomposition that
was constructed according to the likely changes. Section \ref{SecConnection}
specifies the connections between the software requirements and the
modules. Section \ref{SecMD} gives a detailed description of the
modules. Section \ref{SecTM} includes two traceability matrices. One checks the
completeness of the design against the requirements provided in the SRS. The
other shows the relation between anticipated changes and the modules. Section
\ref{SecUse} describes the use relation between modules.

\section{Anticipated and Unlikely Changes} \label{SecChange}

This section lists possible changes to the system. According to the likeliness
of the change, the possible changes are classified into two
categories. Anticipated changes are listed in Section \ref{SecAchange}, and
unlikely changes are listed in Section \ref{SecUchange}.

\wss{These ideas still apply, but rather than separate sections, maybe you could
  just summarize this information in your revised introduction.}

\subsection{Anticipated Changes} \label{SecAchange}

Anticipated changes are the source of the information that is to be hidden
inside the modules. Ideally, changing one of the anticipated changes will only
require changing the one module that hides the associated decision. The approach
adapted here is called design for
change.

\begin{description}
\item[\refstepcounter{acnum} \actheacnum \label{acAlgorithm}:] The algorithm 
used to solve the given IVP.
\item[\refstepcounter{acnum} \actheacnum \label{acImplementation}:] The 
implementation of the algorithms used to solve the IVP.
\item ...
\end{description}

\subsection{Unlikely Changes} \label{SecUchange}

The module design should be as general as possible. However, a general system is
more complex. Sometimes this complexity is not necessary. Fixing some design
decisions at the system architecture stage can simplify the software design. If
these decision should later need to be changed, then many parts of the design
will potentially need to be modified. Hence, it is not intended that these
decisions will be changed.

\begin{description}
\item[\refstepcounter{ucnum} \uctheucnum \label{ucOutput}:] Output format 
(generated code).
\item[\refstepcounter{ucnum} \uctheucnum \label{ucInput}:] The way the driver 
invokes the software.
\item ...
\end{description}

%\section{Module Hierarchy} \label{SecMH}
%
%This section provides an overview of the module design. Modules are summarized
%in a hierarchy decomposed by secrets in Table \ref{TblMH}. The modules listed
%below, which are leaves in the hierarchy tree, are the modules that will
%actually be implemented.
%
%\begin{description}
%\item [\refstepcounter{mnum} \mthemnum \label{mHH}:] Hardware-Hiding Module
%\item ...
%\end{description}
%
%
%\begin{table}[h!]
%\centering
%\begin{tabular}{p{0.3\textwidth} p{0.6\textwidth}}
%\toprule
%\textbf{Level 1} & \textbf{Level 2}\\
%\midrule
%
%{Hardware-Hiding Module} & ~ \\
%\midrule
%
%\multirow{7}{0.3\textwidth}{Behaviour-Hiding Module} & ?\\
%& ?\\
%& ?\\
%& ?\\
%& ?\\
%& ?\\
%& ?\\ 
%& ?\\
%\midrule
%
%\multirow{3}{0.3\textwidth}{Software Decision Module} & {?}\\
%& ?\\
%& ?\\
%\bottomrule
%
%\end{tabular}
%\caption{Module Hierarchy}
%\label{TblMH}
%\end{table}

\section{Binding Times}

This section will demonstrate the difference between binding times via the 
oft-used power example. Rather than dividing the program into modules, a 
decomposition by binding times is given. Such a decomposition is more 
appropriate when multi-stage programming (MSP) is employed, something 
\cite{Parnas1972a} was unaware of at that time, due to Multi-stage programming 
being much newer. 
%In terms of work assignments, perhaps the functions that make up the program 
%could be assigned to different programmers.

Consider the following example for calculating any $x^n$:
\begin{lstlisting}
let square x = x * x
let rec power n x =
  if n = 0 then 1
  else if n mod 2 = 0 then square (power (n/2) x)
  else x * (power (n-1) x)
(* val power : int -> int -> int = <fun> *)
\end{lstlisting}

(The comment below the code shows the reply of the MetaOCaml interactive 
prompt, called the ``toplevel'', which contains the types of the power 
function.)

Assuming $x^7$ is a common operation, it would be useful to define
\begin{lstlisting}
let power7 x = power 7 x
\end{lstlisting}
to give the \emph{partial application} a name and use it throughout the code. 
Partial application means 
that when giving a function some but not all the arguments it expects, one 
creates a new function that only expects the remaining arguments. In the case 
of \lstinline|power7|, the resulting function type will be 
\lstinline|int -> int|.

In MetaOCaml, one can also create a specialized instance of the power function 
for a particular value of $n$. However, in the case of MetaOCaml, one can 
generate code which accepts any $x$ and computes $x^n$. To do this, the
\lstinline|power| function must be annotated such that computations which can 
be performed at generation time are distinguished from those that must be 
performed at run time. Because information such as $n$ is available at 
right away and certain not to change in the future, any computation that 
only depends on $n$ may be performed when the code is being generated. In 
contrast, information such as $x$ is not known at generation time and subject 
to change, so any computation that needs it will have to be postponed until the 
generated code is actually run (when x is given).
\begin{lstlisting}
let rec spower n x =
  if n = 0 then .<1>.
  else if n mod 2 = 0 then .<square .~(spower (n/2) x)>.
  else .<.~x * .~(spower (n-1) x)>.;;
(* val spower : int -> int code -> int code = <fun> *)
\end{lstlisting}
Two annotations, or staging constructs, specific to MetaOCaml appear in the 
code above. These are brackets \lstinline|.< e >.| and escape \lstinline|.~e|. 
Brackets \lstinline|.< e >.| `quasi-quote' the expression e, annotating it as 
computed later. Escape \lstinline|.~e|, which must be used within brackets, 
tells that e is computed now but produces the result for later. That result, 
the code, is inserted (spliced-in) into the containing bracket. The inferred 
type of \lstinline|spower| is different. The 
result is no longer an \lstinline|int|, but \lstinline|int code| -- the code of 
expressions that compute an \lstinline|int|. The type of \lstinline|spower| 
spells out which argument is received now, and which later: the future-stage 
argument has the \lstinline|code| type. To specialize \lstinline|spower| to 7, 
one can define
\begin{lstlisting}
let spower7_code = .<fun x -> .~(spower 7 .<x>.)>.;;
(*
val spower7_code : (int -> int) code = .<
  fun x_1  -> x_1 * ((* CSP square *) (x_1 * ((* CSP square *) (x_1 * 
    1))))>.
*)
\end{lstlisting}
and obtain the code of a function that expects any $x$ and returns $x^7$. Code, 
even of functions, can be printed, which is what MetaOCaml toplevel 
(interactive prompt) just did. The print-out contains so-called cross-stage 
persistent values, or CSP, which `quote' present-stage values such as square to 
be used later. One may think of CSP as references to `external libraries' -- 
only in our case the program acts as a library for the code it generates. If 
square were defined in a separate file such as \texttt{sq.ml}, then its 
occurrence in \lstinline|spower7_code| would have been printed as 
\lstinline|Sq.square|.

To use the specialized $x^7$ now, in our code, we should `run' 
\lstinline|spower7_code|, applying the function 
\lstinline|Runcode.run : 'a code -> 'a|. The function compiles the code and 
links it back to our program. Runcode.run is aliased to the prefix operation 
(!.):
\begin{lstlisting}
open Runcode
let spower7 = !. spower7_code
(* val spower7 : int -> int = <fun> *)
\end{lstlisting}

The specialized \lstinline|spower7| has the same type as the partially applied 
\lstinline|power7| above. They behave identically. However, 
\lstinline|power7 x| will do the recursion on $n$, checking $n$'s parity, etc. 
In contrast, the specialized \lstinline|spower7| has no recursion (as can be 
seen from \lstinline|spower7_code|). All operations on n have been done when 
the \lstinline|spower7_code| was computed, producing the straight-line code 
\lstinline|spower7| that only multiplies $x$.

(MetaOCaml supports an arbitrary number of later stages, letting one write not 
only code generators but also generators of code generators, etc.)

\wss{I like the above summary.  I would like you to use the same systematic
  steps for presenting your code.}

\subsection{Compile-Time}
As the name implies, this is when the code is compiled. Code generation could 
occur at this time, but this will not happen for the RK generator.

The generator and its RK methods will be completely defined at this time. There 
are no inputs to it at this time, because it is effectively a library.

\subsection{Generation Time}
This is part of the run-time of the driver program, but from the point of view 
of this library, separating the moment the code for a particular IVP is 
generated and the moment(s) it is executed is useful.

The ODE, interval, step size, desired RK method and initial value are known at 
generation time. 
Since these will not change when looking for solutions given a particular point 
on the interval for which the algorithm is solving, the IVP can be solved at 
this time.

The type of the ODE is \lstinline|float -> float array -> float array|. The 
types for the range, step size and initial value are 
\lstinline|(float * float)|, \lstinline|float|, and \lstinline|float array| 
respectively.
The type of the function that is returned should be (close to) 
\lstinline|(float -> float) code|.

\subsection{Run-Time}
This is also part of the driver program's run-time, but from the point of view 
of the RK generator, this is (one of) the moment(s) when the generated code is 
run.

The function that will return values for any input within the specified 
interval will execute at run-time.
As shown in the previous subsection, the input will be a \lstinline|float|, and 
the return type is another \lstinline|float|.

\section{Connection Between Requirements and Design} \label{SecConnection}

N/A. While the commonality analysis has clear implications, the requirements 
are not explicitly recorded anywhere. 

\section{Module Decomposition} \label{SecMD}
N/A. The program is broken down by binding times, rather than modules, due to 
the use of multi-stage programming via MetaOCaml.
%
%Modules are decomposed according to the principle of ``information hiding''
%proposed by \citet{ParnasEtAl1984}. The \emph{Secrets} field in a module
%decomposition is a brief statement of the design decision hidden by the
%module. The \emph{Services} field specifies \emph{what} the module will do
%without documenting \emph{how} to do it. For each module, a suggestion for the
%implementing software is given under the \emph{Implemented By} title. If the
%entry is \emph{OS}, this means that the module is provided by the operating
%system or by standard programming language libraries.  Also indicate if the
%module will be implemented specifically for the software.
%
%Only the leaf modules in the
%hierarchy have to be implemented. If a dash (\emph{--}) is shown, this means
%that the module is not a leaf and will not have to be implemented. Whether or
%not this module is implemented depends on the programming language
%selected.
%
%\subsection{Hardware Hiding Modules (\mref{mHH})}
%
%\begin{description}
%\item[Secrets:]The data structure and algorithm used to implement the virtual
%  hardware.
%\item[Services:]Serves as a virtual hardware used by the rest of the
%  system. This module provides the interface between the hardware and the
%  software. So, the system can use it to display outputs or to accept inputs.
%\item[Implemented By:] OS
%\end{description}
%
%\subsection{Behaviour-Hiding Module}
%
%\begin{description}
%\item[Secrets:]The contents of the required behaviours.
%\item[Services:]Includes programs that provide externally visible behaviour of
%  the system as specified in the software requirements specification (SRS)
%  documents. This module serves as a communication layer between the
%  hardware-hiding module and the software decision module. The programs in this
%  module will need to change if there are changes in the SRS.
%\item[Implemented By:] --
%\end{description}
%
%\subsubsection{Input Format Module (\mref{mInput})}
%
%\begin{description}
%\item[Secrets:]The format and structure of the input data.
%\item[Services:]Converts the input data into the data structure used by the
%  input parameters module.
%\item[Implemented By:] [Your Program Name Here]
%\end{description}
%
%\subsubsection{Etc.}
%
%
%\subsection{Software Decision Module}
%
%\begin{description}
%\item[Secrets:] The design decision based on mathematical theorems, physical
%  facts, or programming considerations. The secrets of this module are
%  \emph{not} described in the SRS.
%\item[Services:] Includes data structure and algorithms used in the system that
%  do not provide direct interaction with the user. 
%  % Changes in these modules are more likely to be motivated by a desire to
%  % improve performance than by externally imposed changes.
%\item[Implemented By:] --
%\end{description}
%
%\subsubsection{Etc.}
%
\section{Traceability Matrix} \label{SecTM}
N/A.
%
%This section shows two traceability matrices: between the modules and the
%requirements and between the modules and the anticipated changes.
%
%% the table should use mref, the requirements should be named, use something
%% like fref
%\begin{table}[H]
%\centering
%\begin{tabular}{p{0.2\textwidth} p{0.6\textwidth}}
%\toprule
%\textbf{Req.} & \textbf{Modules}\\
%\midrule
%R1 & \mref{mHH}, \mref{mInput}, \mref{mParams}, \mref{mControl}\\
%R2 & \mref{mInput}, \mref{mParams}\\
%R3 & \mref{mVerify}\\
%R4 & \mref{mOutput}, \mref{mControl}\\
%R5 & \mref{mOutput}, \mref{mODEs}, \mref{mControl}, \mref{mSeqDS}, 
%\mref{mSolver}, \mref{mPlot}\\
%R6 & \mref{mOutput}, \mref{mODEs}, \mref{mControl}, \mref{mSeqDS}, 
%\mref{mSolver}, \mref{mPlot}\\
%R7 & \mref{mOutput}, \mref{mEnergy}, \mref{mControl}, \mref{mSeqDS}, 
%\mref{mPlot}\\
%R8 & \mref{mOutput}, \mref{mEnergy}, \mref{mControl}, \mref{mSeqDS}, 
%\mref{mPlot}\\
%R9 & \mref{mVerifyOut}\\
%R10 & \mref{mOutput}, \mref{mODEs}, \mref{mControl}\\
%R11 & \mref{mOutput}, \mref{mODEs}, \mref{mEnergy}, \mref{mControl}\\
%\bottomrule
%\end{tabular}
%\caption{Trace Between Requirements and Modules}
%\label{TblRT}
%\end{table}
%
%\begin{table}[H]
%\centering
%\begin{tabular}{p{0.2\textwidth} p{0.6\textwidth}}
%\toprule
%\textbf{AC} & \textbf{Modules}\\
%\midrule
%\acref{acHardware} & \mref{mHH}\\
%\acref{acInput} & \mref{mInput}\\
%\acref{acParams} & \mref{mParams}\\
%\acref{acVerify} & \mref{mVerify}\\
%\acref{acOutput} & \mref{mOutput}\\
%\acref{acVerifyOut} & \mref{mVerifyOut}\\
%\acref{acODEs} & \mref{mODEs}\\
%\acref{acEnergy} & \mref{mEnergy}\\
%\acref{acControl} & \mref{mControl}\\
%\acref{acSeqDS} & \mref{mSeqDS}\\
%\acref{acSolver} & \mref{mSolver}\\
%\acref{acPlot} & \mref{mPlot}\\
%\bottomrule
%\end{tabular}
%\caption{Trace Between Anticipated Changes and Modules}
%\label{TblACT}
%\end{table}
%
\section{Use Hierarchy Between Modules} \label{SecUse}
N/A.
%
%In this section, the uses hierarchy between modules is
%provided. \citet{Parnas1978} said of two programs A and B that A {\em uses} B 
%if
%correct execution of B may be necessary for A to complete the task described in
%its specification. That is, A {\em uses} B if there exist situations in which
%the correct functioning of A depends upon the availability of a correct
%implementation of B.  Figure \ref{FigUH} illustrates the use relation between
%the modules. It can be seen that the graph is a directed acyclic graph
%(DAG). Each level of the hierarchy offers a testable and usable subset of the
%system, and modules in the higher level of the hierarchy are essentially 
%simpler
%because they use modules from the lower levels.
%
%\begin{figure}[H]
%\centering
%%\includegraphics[width=0.7\textwidth]{UsesHierarchy.png}
%\caption{Use hierarchy among modules}
%\label{FigUH}
%\end{figure}

%\section*{References}

\bibliographystyle {plainnat}
\bibliography {MG}

\end{document}